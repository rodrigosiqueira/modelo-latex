\chapter{Trabalhos Analisados}
\label{cap:trabalhos-analisados}

\section{Implementação Estrutural Leve}

\subsection{Dune}

\subsection{Shreds}

\section{Implementação Estrutural Pesada}

\subsection{Wedge}

Existe um princípio chamado de \emph{"Princípio do Menor Privilégio"} que diz:

\begin{quote}
Divida o código em compartimentos na qual cada um vai executar com o menor
privilégio para completar a sua tarefa. Essa abordagem previne contra ataques
e também previne bugs que levam ao vazamento de dados.
\citep{protectionprinciple}
\end{quote}

Apesar deste princípio ter sido apresentado em 1975 e ser amplamento aceito,
nota-se que várias aplicações que tem acesso a Internet (um ponto de acesso
para o mundo) não aplicam buscam se atender a este fundamento e acabam abrindo
pottencias pontes de vazamento de dados ou falhas de segurança. Alguns
pesquisadores defendem que esse problema surge do fado de que os SO atuais
garantem privilégios por padrão. Isto significa que impor limites requer mais
esforço por parte do programador e também é propenso a erros.

A chamada de sistama \texttt{fork} ilustra a característica de garantir
privilégios por padrão. Quando o processo pai cria um filho, este recebe uma
cópia praticamente exata ao do pai. Se o programador desejar restringir o
nível de acesso do processo filho é preciso realizar uma série de operações
manuais que não garantem que nenhum aspecto tenha sido ignorado.

\cite{wedge} buscaram resolver tal problema mudando o princípio atual de
garantir prinvilégio por padrão, para negar privilégios por padrão. Nessa
abordagem o programador deve indicar explicitamente que deseja garantir
permissões adicionais. Os autores alteraram o GNU/Linux para que este
suportasse a remoção de permissões por patrão e chamaram o modelo proposto por
eles de \emph{Wedge}. O \emph{Wedge} fornece um conjunto de primitivas de
programação para permitir criar componentes com semântica de negar permissão
por padrão tornando o acesso mais restrito. Para isto, o \emph{Wedge} oferece
um esquema simples e flexível de marcação (\emph{tagging}) da memória o que
permite ao programador alocar objetos de memória distintas com a mesma
marcação.

O \emph{Wedge} permite que o programdor crie um número arbitário de
componentes, cada um sem privilégios por padrão, mas existem mecânismos que
permitem o controle fino. Neste sentido o \emph{Wedge} busca se assemlhar as
primitivas atuais para facilitar a adoção e para que seja simples a introdução
delas em aplicações legadas. A proposta é constituída de três primitivas:
\emph{Sthread}, \emph{Tagged Memory}, and \emph{callgates}. O acesso a todas
essas primitivas é feito via chamada de sistemas.

Uma \emph{Sthread} define um compartimento dentro de uma aplicação, o
programador pode atribuir permissão de acesso para a memória e outros recursos
por meio das \emph{sthreads}. Estas consistem de uma thread de controle e
políticas de segurança associado a si, que especificam:

\begin{itemize}
	\item As marcações de memória que o \emph{sthread} pode acessar e a permissão
				para cada um (leitura, leitura-escrita, copy-on-write);
	\item Os descritores de arquivo que a \emph{sthread} pode acessar e as
				permissões de cada (leitura, escrita, leitura-escrita);
	\item Os \emph{callgates} que a \emph{sthread} pode chamar;
	\item O \emph{Unix user id}, o \emph{root directory} e o \emph{SELinux}.
\end{itemize}

Uma nova \emph{sthread} não mantém as permissões de acesso por padrão, a não
ser pelo \emph{snapshot} intocado do \emph{copy-on-write} acessado. Note que o
novo componente não tem permissão para acessar qualquer outra memória ou mesmo
descritor de arquivo do pai. Contudo o pai pode anexar permissões de acesso no
filho durante a sua criação. O \emph{sthread} só pode criar um filho com
permissão igual ou menor do que a sua. Para a implementação os autores criaram
o \emph{sthread} como uma variante dos processos do GNU/Linux, só que ao invés
de herdar as informações do pai ela apenas herda as regiões de memória e tabela
de descritores assegurada pela política de segurança.

Quando alguma alocação de memória ocorre, o programador deve marcar a memória
com uma única \emph{tag}; as permissões de acesso a está memória são garantidas
em termos destas. Por exemplo, pode-se definir leitura e escrita para uma
memória com a \emph{tag} "t". O programador expressa os privilégios de memória
em termos das \emph{tags} que são atribuídos para a memória em tempo de
alocação. Vale acrescentar que uma \emph{tag} nunca implica em privilégios de
outra \emph{tag}. A operação de criação de \emph{tags} começa pela criação dela
por parte do programador, essa operação aloca um segmento de memória e armazena
um mapa da \emph{tag} para o segmento. Em seguida o programador chama a função
para alocar a memória marcada (\texttt{smalloc}) especificando a \emph{tag} e o
tamanho solicitado e assim aloca um \emph{buffer} do segmento com a \emph{tag}.

Um \emph{callgate} é uma porção de código que executa em um nível de privilégio
(normalmente) diferente de quem o chamou. Como um \emph{sthread} sempre roda
com o menor privilégio possível, sempre que for preciso executar algo em um
nível de permissão maior, será preciso chamar um \emph{callgate} para executar
a operação. Note que instalar um \emph{callgate} com acesso para dados
sensíveis funciona como um isolamento. Um \emph{callgate} define um conjunto de
permissões, um ponto de entrada e argumentos de confiçã definods pelo criador.

\subsection{Resource Container}

\cite{resourcecontainers} argumentam que as abtrações de SO tem recebido grandes
contribuições para melhorar questões de desempenho, mas argumentam que pouca
atenção tem sido dado para o gerenciamento de recursos. \emph{Denial Attack}
consiste em fazer com que todos os recursos da máquina sejam consumidos,
ataques como esses ainda são comums e ilustram como a gerência de recursos é
um elemento crítico nos SOs. Neste sentido, \cite{resourcecontainers}
assumem que a raiz dos problemas de gestão de recursos encontra-se no modelo
atual dos SOs de propósito geral; isto significa que primitivas de
escalonamento e de gerência de recursos extende-se por todo o Kernel. Esse
acoplamento torna complicado o processo de permitir que uma aplicação controle
os seus próprios recursos.

%TODO: Verificar se vale a pena adicionar a figura do problema

Para ilustrar as questões referentes a gestão de recursos nos SOs, a figura X
ilustra 4 cernários diferentes. O primeiro cenário consiste em uma aplicação
rodando em user space que não necessita de qualquer recursos diretamente
controlado pelo Kernel, por tanto, a seu consumo de recursos não se estende
para o Kernel. O segundo cenário representa uma aplicação que precisa utilizar
recursos do Kernel para terminar alguma de suas tarefas (e.g., uma aplicação
que usa recursos de rede), note que parte da aplicação consome recuros dentro
do user space; contudo, a outra parte da aplicação expande para os recuros do
Kernel. O terceiro cenário mostra multiplos processos cooperando para concluir
uma tarefa sem precisar de recursos do Kernel. Por fim, tem uma aplicação com
um único processo que dispõe de multiplas threads que além de utilizar recursos
no user space também demandam recursos oriundos do Kernel. Repare como sistemas
tradicionais tem pouco ou nenhum controle sobre os recursos consumidos, veja
que esses casos podem levar o escalonador a fazer contagens imprecisas levando
o mesmo a comportar-se de forma inadequada.

Inspirados pela falta de uma separação clara do gerenciamento de recursos por
parte dos SOs, \cite{resourcecontainers} sugeriram a criação de uma nova
abstração chamada \emph{Resource Containers (RC)}. Este é uma entidade que
logicamente contém todos os recursos utilizados pela aplicação para atingir uma
determinada atividade. Este tem atributos que são usados para fornecer
parâmetros de escalonamento, limites de recursos, e \emph{QoS} para rede. A
ideia é que o Kernel cuidadosamente manipule esses recursos tal qual CPU e
memória consumido por um RC. O escalonador pode acessar as informações
provenientes dos RCs e utilizar elas para melhorar o processo de escalonamento.

A adição do RC cria uma clara distinção entre o domínio da proteção e o de
recursos, além de forner um controle fino sobre os recursos. O RC fornece um
mecânismo para ligar os recursos a uma thread, isto pode ser feito de forma
dinâmica e sob controle da aplicação (isto também é chamado de \emph{resource
binding}). Note que isto permite que multiplas threads podem ser associadas a
multiplos RC ao mesmo tempo, mas por padrão, uma thread começa com um RC
herdado do processo pai. A aplicação poder religar-se para qualquer outro
container caso exista a necessidade.

Uma das implicações diretas do RC é que uma aplicação associe informações com
uma ativiade ao invés de associar a uma thread ou processo, isto permite que
o \emph{scheduler} forneça informações diretas para a atividade.
% TODO: Voltar no artigo e elaborar mais essa questão dos atributos
A alocação de atributos está diretamente apropriada para um modelo de
escalonamento associado com cada RC em um sistema. Uma thread é normalmente
escalonada de acordo com os atributos do container na qual é limitada. Contudo,
se uma thread é mutiplixada entre vários containers, pode custar muito o
reescalonamento, para cada mudança de binging. Na prática a thread deve ser
escalonada baseado na combinação entre alocação de recursos e uso de
containers. Para isto, o modelo define um \emph{scheduler binging} entre cada
thread e um conjunto de container sob o atual sendo multiplexado. 

É interessante observar que os containers formam hieraquias, ou seja, o uso de
um recurso usado por um container filho é restringido pelos parâmetros do
container pai. Por exemplo, se o container pai tem 70\% dos recursos
garantidos, então os container filhos não podem extrapolar esse limite.

De forma geral, para se utilizar o RC os autores sugerem uma API na qual
contém:
\begin{itemize}
	\item \textbf{Criar um Novo Container:} Por padrão, um container é criado
				como parte do \texttt{fork} e são visíveis para a aplicação na forma
				de descritor de arquivos;
	\item \textbf{Ajuste do Container API:} Um processo pode mudar quem e o seu
				container pai;
	\item \textbf{Liberar o Container:} Um processo pode liberar a sua referência
				para um container usando o \texttt{close};
	\item \textbf{Compartilhar Container entre Processos:} Quando um processo
				recebe uma referência para um container, ele pode usar o recurso como
				um \emph{resource context}. Isto permite mover uma aplicação ou
				computação entre multiplos domínios de proteção;
	\item \textbf{Atributos do Container:} Uma aplicação pode ler e ajustar
				atributos de um container;
	\item \textbf{Uso de Informações dos Containers}
	\item \textbf{Binding threads para um container:} Uma thread pode fazer o
				\emph{binding} para uma \emph{thread} a qualquer momento;
	\item \textbf{Resetar o Scheduler Binding}
	\item \textbf{Binding de Socket ou Arquivo para um Container:} Um processo
				pode fazer o \emph{bind} de um socket ou arquivo para um container,
				assim o consumo subsequente dos recursos do kernel ficam no container.
\end{itemize}

\subsection{Nooks}
% deixar claro que tudo isso é no nível do kernel
\cite{nooks} conduziram uma pesquisa motivada pela busca por
adicionar mais confiabilidade e resiliência aos SOs de produção. O principal
problema que os autores buscam resolver são as constantes falhas geradas pela
expansão dos SOs por meio dos programas que controlam um dispositivo, também
conhecido com \emph{modules} ou \emph{device drivers} (Seção \ref{sec:dd}). O
artigo apresenta diversas evidências relacionadas as falhas geradas
pelas extensões feitas nos SOs e sugere uma nova abordagem chamada
\emph{Nooks}.

\emph{Nooks} é uma nova camada que se interpõe entre o Kernel e as extensões.
Essa camada comporta-se como um subsistema responsável por tratar as operações
passadas via kernel para os módulos e vice-versa; o controle feito sobre cada
parte representa uma camada a mais de verificação que adiciona mais segurança e
confiabilidade para o SO. Para fornecer a separação e controle dentro do SO, o
\emph{Nooks} apresenta o conceito de \emph{lightweight kernel protection
domain} (LKPD). Esta abstração permite isolar partes da memória no nível do
Kernel e atribuir diferentes permissões de leitura e escrita para a região.
Note que todo o controle entre as regiões de memória e a comunicação entre o
kernel com as extensões ocorre no domínio do kernel, não tendo relação direta
com o \emph{userspace}.

O projeto do \emph{Nooks} foi totalmente guiado por dois princípios
fundamentais:
\begin{enumerate}
	\item \emph{Nooks} deve ser resistente a falhas e não tolerante a falhas;
	\item \emph{Nooks} deve ser projetado para evitar erros e não para evitar
				abusos.
\end{enumerate}
Esses dois princípios são importantes para deixar claro a área de atuação desta
técnica e para guiar os três objetivos princípais que o \emph{Nooks} busca
atender:
\begin{enumerate}
	\item \textbf{Isolamento:} \emph{Nooks} precisa ser capaz de isolar as
				extensões presentes no Kernel de forma a evitar falhas, consequentemente
				ele precisa ser capaz de detectar falhas antes que outras partes do
				SO sejam afetadas;
	\item \textbf{Recuperação:} A arquitetura precisa dar suporte para a
				recuperação da aplicação em caso de falhas;
	\item \textbf{Compatibilidade com versões anteriores:} \emph{Nooks} deve ser
				compatível com as extensões já existentes e usadas.
\end{enumerate}

% TODO: FIGURA

A Figura X ilustra de forma geral a arquitetura do \emph{Nooks}. Note que o
"diálogo" entre o Kernel e as extensões (vice-versa) precisam passar pelo
\emph{Nooks}. Esse intermédio é conduzido pelo \emph{Nooks Isolation Manager
(NIM)} que é o responsável por implementar os objetivos do projeto. O
\emph{Nooks} precisa estender-se em parte para o Kernel e em parte para as
extensões. No Kernel, é preciso alterar aquelas funções que são
disponibilizadas para as extensões ou que interagem com os módulos. Felizmente,
este tipo de função pode ser detectada por algum padrão que por sua vez pode
ser abstraído para um \emph{script} e este tipo de modificação precisa ser
feita uma única vez. Do ponto de vista dos módulos, no geral, são poucos os
casos que precisam ser alterados diretamente uma vez que este faz uso dos
recursos fornecido pelo Kernel. Normamente, módulos que exportam estruturas de
dados próprias e que são exportáveis para o Kernel precisam ser alteradas.

O \emph{Nooks} fornece isolamento por meio do LKPD uma vez que toda extensão
adicionada ao SO executa dentro do seu próprio LKPD que por sua vez é chamado
de \emph{contexto de execução}. Os domínios utilizam o mesmo nível de proteção
fornecido pelo processador para o Kernel, com a diferença de que o acesso de
escrita para certas porções é limitada e gerenciada pelo \emph{NIM}. Neste
sentido, podemos dividir o esquema de isolamento do \emph{Nooks} em duas partes:
o gerenciador de memória e o \emph{Extension Procedure Call} (XPC).

% TODO: FIGURA

O gerenciador de memória do \emph{Nooks} é responsável por alocar, desalocar e
manipular os LKPD. A Figura X ilustra o kernel address space, note que o Kernel
tem permissão de escrita e leitura sobre toda a memória, mas os LKPD mostrados
só tem acesso de leitura e escrita a si mesmo. A leitura de outras regiões de
memória de dentro de um LKPD depende do controle de acesso feito pelo NIM, mas
no geral só é permitido a leitura de outras regiões da memória.

O outro mecânismo de isolamento fornecido pelo \emph{Nooks} é o XPC, este atua
como um serviço de transferência de controle criado para isolar as operações
feitas dentro do Kernel. A transferência de controle pode acontecer em duas
vias: do Kernel para a extensão (\texttt{nooks\_driver\_call}) e da extensão
para o Kernel (\texttt{nooks\_kernel\_call}). Basicamente essas funções esperam
três argumentos: um ponteiro para a função que será executada, uma lista de
argumentos e o domínio em que será executada. Toda vez que a rotina de
transferência é chamada o contexto da stack que realizou a chamada é salva e o
\emph{Page Table} para o novo domínio alvo é carregado.

Um dos objetivos do \emph{Nooks} é o de ser usado em SO de produção, para isto
ele tem uma preocupação direta em se integrar com o SO de forma a causar o
menor impacto possível nas extensões. Por isto, o \emph{Nooks} isola as
extensões com o XPC e também implementa um mecânismo de rastreamento de
estruturas de dados chamado \emph{object-tracking} (parte do \emph{NIM}). O
\emph{object-tracking} tem três recursos básicos: manter uma lista de
estruturas de dados do Kernel que são manipulados por uma extensão, controlar
todas as modificações para estas extensões e fornecer informações de objetos
para limpeza de dados quando a extensão falhar. O rastreamento dos objetos
começa com o armazenamento de todos os objetos em uso por uma extensão e em
seguida é feita uma associação entre o Kernel e as versões das extensões.

Por fim, o \emph{Nooks} possuí um mecânismo de recuperação que busca detectar
falhas de software quando uma extensão é chamada de forma inapropriada ou se
uma extensão está consumindo muitos recursos. A estratégia de recuperação é
subdividida em duas partes: \emph{recovery management} e \emph{user-mode
agent}. O primeiro é responsável por desalocar recursos e se necessário
realocar novamente os mesmos. O segundo é quem coordena a ação de recuperação e
é definida no espaço de usuário. Note que a flexibilidade do \emph{agent-mode}
permite várias possíveis formas de recuperação, sendo a mais direta forçar a
remoção da extensão, seguida da reinserção e em seguida prosseguindo com a
execução da aplicação.

O \emph{Nooks} utilizou uma base de dados de inserção de falhas em módulos para
validadar a sua eficácia, conseguindo tratar 99\% dos casos. Além disto, os
autores também validaram o trabalho da perspectiva do desempenho e notaram que
na maioria dos casos os gastos com a estrutura são compensado pelas vantagens.

\subsection{Mondrian Memory Protection e Mondrix}

\cite{mmp} buscaram explorar técnicas para realizar controle fino sobre da
memória, com o objetivo de adicionar mais confiabilidade e segurança para os
SOs. Os autores discutem os problemas associados com a decisão de se promover
isolamento por meio da separação do espaço de endereçamento dos processos, que
por sua vez faz com que todas as \emph{threads} de um processo compartilhem um
mesmo domínio de proteção. Além disto o modelo de paginação adotado tem como
menor nível de controle sobre a memória definido pelo tamanho da página, logo,
o menor compartilhamento possível é o de uma página que tem o mesmo nível de
permissão para todos os elementos contido nela. Os autores argumentam que
apesar das vantagem oferecidas pelo modelo atual, muitos problemas de segurança
e estabilidade do sistema são comprometidos. Dois exemplos são apontados pelos
autores, o primeiro ilustra o caso do Servido Apache que pode carregar
multiplos plugins externos, contudo, se algum código malicioso ou defeituoso
for carregado este comprometerá toda a estabilidade do servidor (em alguns
casos quebrando a aplicação). Outro exemplo refere-se ao acesso de ponteiros em
outros domínios da memória (por exemplo, acesso do espaço de usuário para o do
Kernel) que atualmente não são permitidos sendo necessário realizar operações
de cópias.

% TODO: Figura Y

Os autores defendem que mecânismos de controle fino do acesso à memória podem
trazer grandes benefícios em termos de segurança, confiábilidade e em certos
casos ganho de desempenho. Nesse sentido \cite{mmp} propuseram uma abordagem
que permite o controle de acesso da memória no nível do tamanho das palavras de
dado, para isto é apresentando um novo tipo de abordagem baseada em hardware e
software chamada de \emph{Mondrian Memory Protection (MMP)}~\citep{mmp}. Além
disto uma implementação prática chamada de \emph{Mondrix}~\cite{mondrix} foi
feita no GNU/Linux. A Figura Y, resume a representação de memória que o MMP
busca atingir, repare que não existem porções de memória específica; note que a
memória é dividida em diversos compartimentos chamados de \emph{Protection
Domain (PD)}. Um PD é definido como um contexto que determina a permissão para
executar um código. Cada PD é independente do espaço de endereçamento; cada PD
pode conter múltiplas threads, mas cada thread só pode pertencer a um único PD.
O projeto do MMP é guiado por três requisitos:

\begin{enumerate}
	\item \textbf{Diferente}: Cada PD diferente pode ter um domínio deferente
				para a mesma região de memória;
	\item \textbf{Pequeno}: A granularidade de compartilhamento deve ser menor
				do que uma página;
	\item \textbf{Revogável}: Um PD é dono da sua própria região de memória e
				pode mudar a forma como outros domínios veem a sua memória.
\end{enumerate}

% TODO: FIGURA K

Os autores afirmam que o MMP tem a semântica da segmentação sem os problemas. O
MMP fornece controle fino sobre a proteção dos dados compartilhados, utiliza
endereçamento linear, é compátivel com \emph{ISA} existentes, não necessita de
registradores de segmentos, possui um mecânismo simples de revogação de
permissões e não tem ponteiros marcados. A Figura K mostra o comparativo entre
a abordagem utilizando segmentos e o MMP. Para que o MMP possa garantir o
controle de acesso à memória este precisa verificar as permissões de acesso
feitas por cada operação de \emph{load/store}, isto claramente consome mais
recursos computacionais. Por isto o MMP é uma solução que também depende da
utilização de um hardware personalizado.

% TODO: FIGURA G

A Figura G ilustra a arquitetura de hardware proposta para a implementação do
MMP, note que ela introduz muitos elementos novos. %TODO: Descrever o hardware

Da perpectiva do Software, o Mondrix \citep{mondrix} sugere a utilização de um
novo elemento chamado de Supervisor de Memória, este é dividido em duas partes:
parte superior responsável pelo controle do acesso; parte inferior responsável
por escrever nas tabelas de permissão. A parte superior é a mais importante,
sendo responsável por manter e controlar as políticas, interface com o Kernel,
rastrear objetos compartilhados e implementar grupos de permissão. As
definições de permissão da memória é descrita por três elementos:
\begin{enumerate}
	\item \emph{Permissão de Acesso}: As permissões de acesso são definidas pelo
				seguinte conjunto de elementos: permissão do domínio, \emph{gates},
        tabela de \emph{stack};
	\item \emph{Dono da Memória}: Define quem tem autoridade sobre a região de
				memória (domínio). Note que o espaço de endereçamento é contínuo e sem
				sobreposição, logo, cada região tem apenas um dono;
	\item \emph{Permissões Exportadas}: É possível exportar as permissões de
				acesso, isto é feito para permitir que outras regiões chamem códigos
				fora do seu domínio. Contudo a exportação pode ser controlada de forma
				que outros trechos que acessem o código, tenham uma visão limitada da
				região.
\end{enumerate}
Dado esses elementos, o Supervisor de memória atual tomando as decisões
referentes ao acesso a uma dada região da memória.

Um aspecto interessante de se notar do Supervisor de Memória é a sua separação
lógica do alocador de memória. O Mondrix permite que o Kernel escolha qual
alocador de memória deseja utilizar, isto é possível pois um domínio solicita
memória para o alocador e o Supervisor só estabelece as permissões para a
memória solicitada.
% Gosto muito dessa ideia de separar o alocador do controlador de acesso à
% memória, na minha opinião em termos práticas essa é uma das ideias mais
% valiosas desse trabalho.

O Supervisor também é responsável por revogar permissões quando uma região de
memória é liberada, além de manter o rastreamento de qual domínio tem acesso a
qual região de memória. O Supervisor também gerência o acesso a \emph{stack}
das \emph{threads} (uma \emph{thread} só pode controlar \emph{stacks} que estão
no seu domínio), cria e destrói proteções de domínio e também valida as
políticas de acesso.

\subsection{Light-weight Contexts}

\subsection{SpaceJMP}

\subsection{Resource Containers}

\section{Implementação Independente}

\subsection{Exokernel}

\cite{exokernel} defendem que os projetos de SOs possuem uma série de
desvantagens por terem várias abstrações diretamente programada no núcleo. Os
autores argumentam que as abstrações implementadas diretamente no SO possuem
sérios impactos no desempenho, reduz considerávelmente a flexibilidade das
aplicações e reduz as possibilidades de ampliar as funcionalidade
disponibilizadas no espaço do usuário. A implementação dessas abstrações
diretamenta no núcleo do SO são problemática uma vez que: impedem a aplicação
de tirar proveito das otimizações que são específcas do seu domínio, as
mudanças em tais abstrações são desencorajadas e reduzem as possibilidade do
que pode ser feito no espaço de usuário.

Segundo os autores, a fixação das abstrações de alto-nível apresentam as
seguintes limitações:
\begin{itemize}
  \item A degradação do desempenho ocorre uma vez que atender ao maior número
        de aplicações possível com a mesma abstração tem um custo
        computacional. Por exemplo, algumas aplicações podem sofrer com o
        \emph{overhead} gerado por outras camadas de abstrações na qual não
        precisa;
  \item As informações de utilização de recursos são escondidas das aplicações,
        isto dificulta a aplicação ter controle sobre os recuros utilizados;
  \item As aplicações ficam limitadas a utilizar apenas aquelas interfaces
        padrão fornecidade pelo SO.
\end{itemize}
Por fim, com esses problemas em mente e motivados pela antiga observação de que:
quanto mais baixo o nível das primitivas (simples), mais eficiêntemente ela
pode ser implementada e maior a amplitude de ação fornecida para abstrações de
alto-nível; os propuseram uma nova arquitetura de SO chamada \emph{exokernel}.

\emph{Exokernel} busca ser um núcleo mínimo que permita multiplexar de forma
segura os recursos de hardware e ao mesmo tempo fornecer uma interface de
baixo-nível na qual as abstrações do SO podem ser construída sobre ela. Para que
seja possível implementar um SO usando este conceito é preciso implementar a
chamada biblioteca do SO com acesso aos recursos de baixo nível e com as
abstrações construídas nela (e.g. processos, arquivos, escalonador, etc). Para
que a implementação de tal conceito seja possível, os autores os seguintes
princípios:
\begin{enumerate}
  \item \textbf{\emph{Expor o hardware de forma segura:}} O conceito
        fundamental do \emph{exokernel} consistem em exportar de forma segura
        e controlada o acesso de baixo nível ao hardware, por isto toda
        implementação do \emph{exokernel} deve concentrar-se em exportar os
        privilégios e recursos de máquina. Por isto ele não deve impor
        abstrações de alto-nível, i.e, o \emph{exokernel} deve evitar gerir
        recursos;
  \item \textbf{\emph{Expor alocações:}} Um \emph{exokernel} deve permitir que
        uma biblioteca de SO aloque recursos. Além disto, os recursos não devem
        ser implicitamente alocados uma vez que a bibliteca de SO tem a
        obrigação de participar de todas as decisões de alocação;
  \item \textbf{\emph{Expor nomes:}} Um \emph{exokernel} deve exportar nomes
        físicos. Isto é eficiênte uma vez que reduz o nível de indireção, e por
        sua vez o número de traduções necessárias entre nomes virtuais e
        fisícos. Por fim expor nomes físicos também encapsula atributos de
        recuros uteis, por exemplo, um sistema de caches fisicamente indexado
        e diretamente mapeado, tem o nome da página física (i.e. o número da
        página) determina qual página que conflita;
  \item \textbf{\emph{Expor revogações:}} Um \emph{exokernel} deve utilizar um
        protocolo de revogação de recursos explicitas para que a biblioteca do
        SO pssa gerir os recursos de forma eficiênte;
\end{enumerate}
Por fim, além desses princípios o \emph{exokernel} também deve especificar as
políticas para arbitrar a competição entre as bibliotecas do SO. Dado esses
princípios o \emph{exokernel} deve disponibilizar as seguinte tarefas: ligação
segura, visibilidade de recuros e protocolo de falha.

A ligação segura (\emph{Secure Bindings - SB}) é o mecânismo que desacopla a
autorização do uso dos recursos. Para implementar o SB é preciso um conjunto de
primitivas que a aplicação pode usar para expressar verifcações de proteção.
Essas podem ser implementadas em hardware ou software, por exemplo, a entrada
da TLB é uma primitiva de hardware. Na prática o SB trabalha com o conceito de
"baixar o código para o kernel", esse código é chamado para cada acesso ou
evento do recurso para determinar o dono e a ação que o Kernel deve tomar.
Baixar o código dentro do kernel permite uma thread da aplicação ter controle
sobre os eventos. Isto melhora o desempenho uma vez que elimina a camada
intermédiaria do Kernel e também propícia limitar a aplicação aos seus
recursos.

A tarefa de \emph{revogação visível} é o mecânismo para recuperar os recursos
e romper com o SB estabelicido. Vale observar que este mecânismo pode ser
visível ou invisível para a aplicação. Tradicionalmente os SOs realizam a
revogação de forma invisível desalocando recursos sem o involvimento da
aplicação. O \emph{exokernel} utiliza revogação visível para a maioria dos
recursos, msmo o processador é explicitamente revogado ao fim de um tempo de
execução determinado de forma que a biblioteca do SO é informada para que possa
reagir. O processo de revogação é como um diálogo entre o \emph{exokernel} e a
biblioteca do SO; a biblioteca precisa organizar uma lista de recursos que pode
ser desalocado rápidamente.

Por fim, um \emph{exokernel} pode definir um segundo estágio de revogação no
seu protocolo, na qual um pedido se torna um imperativo. Isto ajuda no caso em
que a biblioteca do SO falha, então o \emph{exokernel} pode simplesmente
quebrar o SB e informar a biblioteca. Todos as perdas de recursos forçadas são
armazenadas em um vetor de recuperação.

%TODO: Falta dar o fechamento com a parte dos processos e talz

\subsection{Singularity}

\subsection{Corey}
