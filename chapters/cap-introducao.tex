%% ------------------------------------------------------------------------- %%
\chapter{Introdução}
\label{cap:introducao}

Sistemas Operacionais (SO) são projetos de software considerados grandes e
complexos em que cada uma das suas partes é estritamente definida baseada nas
entradas, saídas e funcionalidades esperadas \citep{silberschatz}.
Coletivamente, essas partes representam um conjunto de abstrações construídas
sobre o hardware (e.g., processadores, memórias e dispositivos de
armazenamento) para fornecer recursos para os usuários --- tanto
desenvolvedores de aplicações quanto usuários finais. Dentre os conceitos
empregados pelos SOs modernos destacam-se processos, escalonadores,
gerenciadores de memória, sistemas de arquivo~\citep{tanenbaum}.

Uma das abstrações mais antigas fornecida pelos SOs é o conceito de
\textit{processos}. Estes dão a noção de virtualização do processador, pois
oferecem a ilusão de múltiplos programas executando
simultaneamente~\citep{love, tanenbaum}. Os SOs de proposito geral possuem
grande dependência sobre o conceito de processos, sendo esse o principal
mecanismo de interação com os demais recursos oferecidos. Dado o papel central
de tal abstração nos SOs modernos, fica evidente que alterações nessa área
reverberam em todo tipo de usuário. Por esse motivo é comum observar que os
processos evoluem de forma a atender novas demandas de hardware, segurança,
escalabilidade, modelos de programação e desempenho.

Infelizmente, melhorar a abstração de processos para atender novos requisitos é
uma tarefa delicada. Em um SO voltado para produção (e.g., GNU/Linux e
FreeBSD), uma mudança incorreta na abstração de processos pode comprometer
completamente a estabilidade do SO e, consequentemente, afetar diversos sistemas em
todo o mundo. Assim, identificar e implementar novos atributos dentro dos
processos normalmente demanda muito esforço e, portanto, raramente esses são
adotados fora de alguns poucos contextos na comunidade científica. Dessa
maneira, para tornar as pesquisas nessa área mais robustas e com maior
potencial de adoção por parte dos SOs mais amplamente utilizados, novas propostas tipicamente
precisam abordar três aspectos: apresentar uma implementação viável,
medir o desempenho e demonstrar benefícios para as aplicações.

Inspirados pelas diversas propostas de novas extensões nas abstrações de
processos sugeridas pela academia nas últimas décadas, nós buscamos
investigá-las com o objetivo de compilar o estado da arte verificando
características interessantes para os SOs usados em produção. Dado o foco
pretendido nas abstrações relativas a processos, examinamos as diferentes
pesquisas excluindo outros aspectos eventualmente introduzido pelos
pesquisadores.

\section{Avaliação de Novas Abstrações de Processos}

Dentre o conjunto dos diversos trabalhos na área de processos é possível
realizar a categorização dos mesmos de acordo com a abordagem adotada. Além do
mais, essas pesquisas normalmente se auto-avaliam em duas fases: (1)
microbenchmark\footnote{É uma maneira de identificar tarefas da aplicação e
medir o desempenho delas~\citep{micro}} e (2) utilizando uma aplicação como
exemplo. Nesse contexto, podemos avaliar de forma direta uma proposta baseada
na sua categoria e de forma mais profunda verificar como ela se auto-avalia.

De forma a rapidamente explorar as característica de uma abordagem,
classificamos as pesquisas em três categorias:

\begin{description}
\item [Implementação Estrutural Pesada]

consiste em melhorar um SO bem estabelecido e amplamente utilizado (e.g.,
GNU/Linux), \textbf{através de modificações em seu núcleo};

\item [Implementação Estrutural Leve]

consiste na implementação das novas abstrações em um sistema já consolidado,
mas \textbf{evitando mudanças diretas no núcleo};

\item [Implementação Independente]

consiste na criação de um SO totalmente novo. Normalmente este tipo de
trabalho introduz inovações em diferentes áreas, contudo, neste texto estamos
interessados nas propostas relacionadas às abstrações de processos.

\end{description}

Essa categorização foi feita com o intuito de facilmente fornecer pistas das
consequências de cada proposta em termos do impacto no projeto de um SO. Cada
uma dessas abordagens oferece um balanceamento entre o escopo das propostas de
modificação, o esforço necessário para o desenvolvimento e a facilidade em
integrar com SOs de uso cotidiano.

Depois que uma implementação se torna funcional, os pesquisadores comumente
utilizam um conjunto de \emph{microbenchmarks} para avaliar o desempenho das
suas propostas. A peculiaridade de cada trabalho é evidenciada na grande
pluralidade de estratégias de \emph{microbenchmarks} encontrada em diversos
artigos.  De forma geral, os cientistas selecionam um conjunto de procedimentos
para mostrar os prós e contras da sua abordagem. Contudo, identificar um bom
conjunto de \emph{microbenchmarks} que comprove um comportamento particular é
uma tarefa que requer certa experiência por parte do pesquisador.

O uso de \textit{microbenchmarks} como forma de comprovação de algum resultado
serve como indicador parcial de bons resultados mas pode ser perigoso, uma vez
que pode induzir o pesquisador a acreditar em um resultado positivo
negligenciando outros potenciais impactos. Assim, para validar as propostas e
verificar o impacto delas em um cenário minimamente realista, os pesquisadores
comumente selecionam uma aplicação utilizada em certos contextos para testar as
modificações feitas. No entanto, selecionar um software para realização de
testes representa uma tarefa complicada devido às características singulares de
cada aplicação. Por exemplo, um software pode utilizar muita memória e consumir
boa parte dos recursos da CPU, o que é desejável para demonstrar o uso de novas
abstrações de processos em uma situação de estresse. Ainda assim, pode não
revelar nada de útil relacionado a melhorias de segurança.

\section{Objetivos}

No cenário atual notamos uma falta de unificação e validação das diferentes
propostas de mudanças nas abstrações dos processos, aspectos esses que são
importantes para justificar sua implementação em um SO de produção. Neste
trabalho, temos como objetivo avaliar iniciativas que melhoram as abstrações de
processos levando em consideração a implementação, as questões que a nova
proposta visa resolver e os impactos gerais no SO. Fizemos essa avaliação por
meio da análise de 15 trabalhos, onde buscamos responder as seguintes
perguntas:

\begin{quote}
 \item \textit{RQ1:.} ``Quais são as características desejáveis para a próxima geração de abstrações de processos?''
 \item \textit{RQ2:.} ``Quais são os principais desafios em se implementar a próxima geração de abstrações de processos?''
 \item \textit{RQ3:.} ``Quais aplicações podem ser utilizadas para avaliar as nova abstração adicionadas ao SO?''
 \item \textit{RQ4:.} ``Qual conjunto de \emph{microbenchmark} pode ser utilizado para auxiliar a entender os impactos de uma nova característica adicionada para as abstrações de processos?''
\end{quote}

Dos 15 trabalhos que analisamos, a maior parte faz uso de SOs baseados no
Kernel Linux. Por isso, no Capítulo~\ref{cap:fundamentacao} exploramos
conceitos gerais sobre SO e também descrevemos alguns desse conceitos usando o
GNU/Linux como exemplo. Para responder a questão RQ1, identificamos as
principais características propostas por vários pesquisadores e extraímos o
conceito central desses vários trabalhos, descritos no Capítulo
\ref{cap:trabalhos-analisados} e
\ref{cap:analise-sobre-abstracoes-de-processos}. Para responder a RQ2,
investigamos diferentes técnicas de implementação levando-se em consideração a
categorização feita sobre as abordagens descritas no Capítulo
\ref{cap:trabalhos-analisados}. Para responder a RQ3, extraímos dos trabalhos
uma coleção de programas que podem se beneficiar das modificações, levando
vantagens para o usuário final (Capítulo \ref{cap:validacoes}). Para responder
a RQ4, nós analisamos diversos \emph{microbenchmarks} utilizados pelos
pesquisadores e selecionamos um subconjuntos de características uteis para
avaliar uma proposta (Capítulo \ref{cap:validacoes}). Por fim, no Capítulo
\ref{cap:analise-sobre-abstracoes-de-processos}, discutimos um novo horizonte
para a próxima geração de abstrações de processos com o objetivo de auxiliar na
construção de um novo modelo.
