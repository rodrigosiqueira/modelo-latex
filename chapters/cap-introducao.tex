%% ------------------------------------------------------------------------- %%
\chapter{Introdução}
\label{cap:introducao}

Sistemas Operacionais (SO) são projetos de software considerados grandes e
complexos, cada uma das suas partes é perfeitamente definida baseada nas
entradas, saídas e funcionalidades \citep{silberschatz}. Coletivamente, essas
partes representam um conjunto de abstrações construídas sobre o hardware
(e.g., processadores, memórias e dispositivos de armazenamento) para fornecer
recursos para os usuários. Alguns dos conceitos empregados pelos SOs modernos
são: processos, escalonadores, gerenciadores de memória, sistemas de arquivo,
dentre outros.  Esses oferecem mecanismos para que as aplicações de usuários
utilizem recursos de hardware.

Uma das abstrações mais antigas fornecida pelos SOs é o conceito de
\textit{processos}. Este dá a noção de virtualização do processador, pois
oferecendo a ilusão de múltiplos programas executando
simultaneamente~\citep{love, tanenbaum}.  Essa abstração evolui continuamente
para atender as demandas de hardware, segurança, escalabilidade, modelos de
programação e desempenho.

Infelizmente, melhorar a abstração de processos para atender novos requisitos é
uma tarefa delicada. Em um SO voltado para produção (e.g., GNU/Linux e
FreeBSD), uma mudança incorreta na abstração de processos pode comprometer
completamente a estabilidade do sistema. Portanto, identificar e implementar
novos atributos dentro dos processos normalmente demanda um significante
esforço; consequentemente esses raramente são adotados fora da comunidade
científica. Dessa maneira, para tornar as pesquisas nessa área mais robustas e
com maior potencial de adoção por parte dos principais SOs, novas propostas
tipicamente precisarão/precisam abordar três aspectos: apresentar uma
implementação viável, medir o desempenho e demonstrar benefícios para as
aplicações.

Iniciativas para melhorar as abstrações de processos vem sendo sugeridas por
diversos grupos de pesquisas, cada uma apresentando uma nova abordagem e
diferentes propostas. Nós classificamos as abordagens utilizadas em uma das
três categorias:

\begin{itemize}
\item \textbf{Implementação Estrutural Pesada} consiste na modificação do núcleo
  de um SO de produção conhecido de forma a demonstrar as melhorias;
\item \textbf{Implementação Estrutural Leve} consiste na implementação das
  novas abstrações em um sistema já consolidado, mas evitando mudanças diretas
  no núcleo;
\item \textbf{Implementação Independente} consiste na criação de um SO
  totalmente novo e sem qualquer dependência. Normalmente este tipo de trabalho
  introduz várias inovações em diferentes áreas, contudo, neste texto estamos
  interessados nas propostas relacionadas as abstrações de processos;
\end{itemize}

Essa categorização foi feita meramente para fornecer pistas das consequências
de cada proposta em termos do impacto no projeto de um SO. Cada uma dessas
abordagens oferece um balanceamento entre o escopo das propostas de
modificação, o esforço necessário para o desenvolvimento e a facilidade em
integrar com SOs de uso cotidiano. Vale observar de que não existe uma
abordagem considerada adequada para todos os projetos de pesquisa.

Depois que uma implementação se torna funcional, os pesquisadores comumente
utilizam um conjunto de \emph{microbenchmark} para avaliar o desempenho das
propostas deles. A peculiaridade de cada trabalho explica a pluralidade de
estratégias de \emph{microbenchmark}: os cientistas selecionam um conjunto de
procedimentos para mostrar os prós e contras da sua abordagem.  Contudo,
identificar o \emph{microbenchmark} para evidenciar um comportamento particular
é uma tarefa que requer certa experiência por parte do pesquisador.

Por fim, para validar as propostas e verificar o impacto delas em um cenário
minimamente realista, os pesquisadores normalmente selecionam uma aplicação
utilizada em certos contextos para testar as modificações feitas. Contudo,
selecionar um software para realizar os testes representa uma tarefa complicada
devido a característica singular de cada aplicação. Por exemplo, um software
pode utilizar muita memória e consumir boa parte dos recursos da CPU, o que é
desejável para demonstrar o uso de novas abstrações de processos em uma
situação de estresse. Ainda assim, pode não revelar nada de útil relacionada a
melhorias de segurança.

Apesar dos diversos trabalhos na área, propostas para expandir os conceitos de
processos ainda necessitam de unificação e validação para justificar o trabalho
de implementar as mesmas em um SO de produção. Neste trabalho, nós avaliamos
iniciativas que melhoram as abstrações de processos levando em consideração
a implementação, as questões que a nova proposta visa resolver e os impactos
gerais no SO. Para isto, nós examinamos 15 trabalhos com o objetivo de
responder as seguintes perguntas:

\begin{quote}
 \item \textit{RQ1:.} "Quais são as características desejáveis para a próxima geração de abstrações de processos?"
 \item \textit{RQ2:.} "Quais são os principais desafios em se implementar a próxima geração de abstrações de processos?"
 \item \textit{RQ3:.} "Quais aplicações podem ser utilizadas para avaliar as nova abstração adicionadas ao SO?"
 \item \textit{RQ4:.} "Qual conjunto de \emph{microbenchmark} pode ser utilizado para auxiliar a entender os impactos de uma nova característica adicionada para as abstrações de processos?"
\end{quote}

Para responder a questão RQ1, nós identificamos as principais características
propostas por vários pesquisadores, extraímos o conceito central dos vários
trabalhos, como descrito no Capítulo \ref{cap:trabalhos-analisados} e
\ref{cap:analise-sobre-abstracoes-de-processos}. Para responder RQ2, nós
investigamos diferentes técnicas de implementação, descritas no Capítulo
\ref{cap:trababalhos-analisados}. Para responder RQ3, nós extraímos dos
trabalhos analisados uma coleção de programas que podem se beneficiar das
modificações levando vantagens para o usuário final (Capítulo
\ref{cap:validacoes}). Por fim, para responder RQ4, nós analisamos diversos
\emph{microbenchmark} adotados pelos pesquisadores e discutimos um subconjuntos
derivado deles (Capítulo \ref{cap:validacoes}).

Combinando o resultado das quatro perguntas de pesquisa, nós derivamos um
conjunto de propriedades desejáveis para a próxima geração de abstrações de
processos. Nos esperamos que atualizações nós conceitos de processos serão
extremamente importante para a discussão sobre a alteração de projeto de SO.
Nos também discutimos as implicações de novas abstrações de processos no
projeto de um SO no Capítulo \ref{cap:analise-sobre-abstracoes-de-processos}.
Este trabalho ajudará a entender as potenciais abstrações de processos e como
validar as mesmas.
