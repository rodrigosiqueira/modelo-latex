%% ------------------------------------------------------------------------- %%
\chapter{Introdução}
\label{cap:introducao}

Sistemas Operacionais (SO) são projetos de software considerados grandes e
complexos; cada uma das suas partes é perfeitamente definida baseada nas
entradas, saídas e funcionalidades \cite{silberschatz}. Coletivamente, essas
partes representam uma coleção de abstrações construídas sobre o hardware
(e.g., processadores, memórias e dispositivos de armazenamento) para fornecer
recursos para os usuários. Alguns dos conceitos empregados pelos SOs modernos
são: processos, escalonadores, gerenciadores de memória e sistemas de arquivo
que oferecem mecanismos para aplicações de usuários utilizem recursos de
hardware.

Uma das abstrações mais antigas fornecida pelos SOs é o conceito de
\emph{processos}. Este da a noção de processamento virtualização, oferecendo a
ilusão de múltiplos programas rodando ao mesmo tempo\cite{love, tanenbaum}.
Esta abstração continuamente evolui para atender as demandas de hardware,
segurança, escalabilidade, modelos de programação e desempenho.

Infelizmente, melhorar a abstração de processos para atender novos requisitos é
uma tarefa delicada. Em um SO voltado para produção (e.g., GNU/Linux e
FreeBSD), uma mudança incorreta na abstração de processos pode comprometer
completamente a estabilidade do sistema. Portanto, identificar e implementar
novos atributos dentro dos processos normalmente demanda um significante
esforço e raramente esses terminam sendo adotados fora da comunidade
científica. Dessa maneira, para tornar as pesquisas nessa área mais robusta e
com maior potencial de adoção por parte dos principais SOs, novas propostas
tipicamente precisão endereçar três aspectos: apresentar uma implementação
viável, medir o desempenho e demonstrar benefícios para as aplicações.

Iniciativas para melhorar as abstrações de processos vem sendo propostas por
diversos grupos de pesquisas, cada um apresentando uma implementação utilizando
propostas diferentes. Nos classificamos as abordagens utilizadas em uma das
três categorias descritas abaixo:

\begin{itemize}
\item \textbf{Implementação Estrutural Pesada} consiste na modificação do núcleo
  de um SO de produção conhecido de forma a demonstrar as melhorias;
\item \textbf{Implementação Estrutural Level} consiste na implementação das
  novas abstrações em um sistema já consolidado, mas evitando mudanças diretas
  no núcleo;
\item \textbf{Implementação Independente} consiste na criação de um SO
  totalmente novo e sem qualquer dependência. Normalmente este tipo de trabalho
  introduz várias inovações em diferentes áreas, contudo, neste texto estamos
  interessados nas propostas relacionadas as abstrações de processos;
\end{itemize}

Essa categorização foi feita meramente para fornecer pistas das consequências
de cada proposta em termos do impacto no projeto de um SO. Cada uma dessas
abordagens oferece um balanceamento entre o escopo das propostas de
modificação, o esforço necessário para o desenvolvimento e a fácil integração
com um SO existente. Vale observar de que não existe uma abordagem considerada
adequada para todos os projetos de pesquisa.

Depois que uma implementação se torna funcional, os pesquisadores comumente
utilizam um conjunto de \emph{microbenchmark} para avaliar o desempenho das
propostas deles. A peculiaridade de cada trabalho explica a pluralidade de
estratégias de \emph{microbenchmark}: os cientistas tem que selecionar um
conjunto de procedimentos para mostra os pros e contras da sua abordagem.
Contudo, identificar o \emph{microbenchmark} para evidenciar um comportamento
particular é uma tarefa difícil e que requer certa experiência por parte do
pesquisador.

Por fim , para validar as propostas e verificar o impacto delas em um cenário
minimamente realista, os pesquisadores normalmente selecionam uma aplicação
utilizada em certos contextos para testar as modificações feitas. Contudo,
selecionar um software para realizar os testes representa uma tarefa complicada
devido a característica singular de cada aplicação. Por exemplo, um software
pode utilizar muita memória e consumir boa parte dos recursos da CPU, o que é
desejável para demonstrar o uso de novas abstrações de processos em uma
situação de pressão. Ainda assim, pode não revelar nada de útil relacionada a
melhorias de segurança.

% "unification"?
Apesar dos diversos trabalhos na área, propostas para expandir os conceitos de
processos ainda necessitam de unificação e validação para justificar o trabalho
de implementar as mesmas em um SO de produção. Neste trabalho, nos avaliamos
iniciativas que melhoram as abstrações de processos levando-se em consideração
a implementação, as questões que a nova proposta visa resolver e os impactos
gerais no SO. Para isto, nos examinamos X trabalhos de pesquisa com o objetivo
de responder as seguintes perguntas de pesquisa:

\begin{quote}
 \item \textit{RQ1:.} "Quais são as características desejáveis para a próxima geração de abstrações de processos?"
 \item \textit{RQ2:.} "Quais são os principais desafios em se implementar a próxima geração de abstrações de processos?"
 \item \textit{RQ3:.} "Qual o conjunto de \emph{microbenchmark} que pode ser utilizado para melhor avaliar os impactos de uma nova característica adicionada para as abstrações de processos?"
 \item \textit{RQ4:.} "Quais aplicações podem tirar o melhor proveito da nova abstração adicionada ao SO?"
\end{quote}

Para responder a questão RQ1, nos identificamos as principais características
propostas por vários pesquisadores, extraímos o conceito central dos vários
trabalhos, como descrito na Seção \ref{cap:trabalhos-analisados} e Seção
\ref{cap:analise-sobre-abstracoes-de-processos}. Para responder RQ2, nos
investigamos diferentes técnicas de implementação, descritas na seção \ref{}.
Para responder RQ3, nos analisamos diversos \emph{microbenchmark} adotados
pelos pesquisadores e selecionamos um conjunto de experimentos (Seção \ref{}).
Por fim, para responder RQ4, nos extraímos dos trabalhos analisados uma coleção
de programas que podem se beneficiar das modificações levando vantagens para o
usuário final (Seção \ref{}).

Combinando o resultado das quatro perguntas de pesquisa, nos derivamos um
conjunto de propriedades desejáveis para a próxima geração de abstrações de
processos. Nos esperamos que atualizações nos conceitos de processos serão
extremamente importante para a discussão sobre a alteração de projeto de SO.
Nos também discutimos as implicações de novas abstrações de processos no
projeto de um SO na Seção \ref{}. Este trabalho ajudará a entender as potências
abstrações de processos e como validar as mesmas.
