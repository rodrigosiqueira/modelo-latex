%% ------------------------------------------------------------------------- %%
\chapter{Conclusões}
\label{cap:conclusoes}

Ao longo desse trabalho investigamos diversos aspectos referentes as abstrações
de processos, começamos revisitando conceitos já consolidados (Capítulo
\ref{cap:fundamentacao}) para que em seguida pudéssemos expandir a visão sobre
os processos, finalmente apresentamos diferentes perspectivas sob tal conceito.
Buscamos o estado da arte e suas intersecções com o estado da prática
utilizando a implementação do MVAS como estudo de caso (Capítulo
\ref{cap:estudo-de-caso}). Durante esse trabalho interagimos com a indústria
por meio do Kernel Linux e com uma parceria com a HPe (tal parceria durou
apenas alguns meses). Por várias vezes tentamos consertar o MVAS não atingindo
exito em tal tarefa, contudo, extraímos dessa experiência diversas ideias na
qual buscamos refina-las realizando uma análise crítica sobre o estado da arte.

Ao nós aprofundarmos na pesquisa sobre novas abstrações de processos percebemos
que por vários anos a academia vem buscando inovações nesse campo, mas tais
progressos não chegam ao estado da prática. Como o Capítulo
\ref{cap:estudo-de-caso} discute, parte disso vem do fato de que algumas
propostas negligenciam aspectos importantes que precisam ser considerados para
avançar o estado da prática. Por exemplo, é fundamental que as novas gerações
de abstração de processos levem em consideração aplicações amplamente adotadas.
No Capítulo \ref{cap:validacoes} evidenciamos um subconjunto de aplicações que
podem servir como alicerce para demonstrar o uso e potencial de adoção de uma
nova abstração. Além disto, não basta levar ganhos em uma área ao custo de
fazer com que outras partes da aplicação tenha o seu comportamento afetado
negativamente.

As validações de novas abstrações também devem passar por um rigoroso processo
de validação para se provar útil. Como exemplo, os estudos de caso revelaram
como uma determinada proposta pode comprometer todo o sistema. Muitos dos
trabalhos analisados não realizam testes de estresse consistentes, o que não
revela a real natureza dos impactos.

Por outro lado, notamos da análise dos diversos trabalhos que não há uma
tentativa de unificar as propostas, normalmente cada trabalho busca
apresentar-se como diferentes de todos os outros já feito. Após o estudo das
diversas propostas, notamos que a melhor forma de levar os avanços sugeridos
pelos pesquisadores para os SOs de uso cotidiano é buscando unificar as
pesquisas e adaptando elas para as necessidades atuais. No Capítulo
\ref{cap:analise-sobre-abstracoes-de-processos} buscamos mostrar tal aspecto
apontando um caminho para a próxima geração de abstrações de processos.
Qualquer tentativa de avançar o estado da prática também deve buscar o menor
impacto possível nos SOs. O conjunto de experimentos realizado, em junção com
os diversos trabalhos apresentados ilustram um potencial avanço nos SOs atuais.

\section{Trabalhos Futuros}

% <chaws> acho que um ponto de partida seria pegar as CVEs existentes de kernel e tentar prioriza-las pra iniciar os testes

Esse trabalho teve como objetivo estudar e por em evidência um campo de
pesquisa com pouca visibilidade. Dentre os trabalhos futuros que identificamos,
nós destacamos: escrever um \textit{survey} sobre o estado da arte das
abstrações de processos e outro criando uma taxonomia das aplicações, a criação
de um sistema de validações de abstrações de processos que funcione de forma
automatizada e a implementação da abstração de processos da próxima geração
proposta nesse trabalho.

Como podemos observar no Capítulo \ref{cap:trabalhos-analisados}, diversos
pesquisadores vem colocando esforços para fazer com que as abstrações de
processo evoluam. Além disto, as abstrações de processos adotadas na prática
também distinguem-se daquelas apresentadas na bibliografia padrão. Por esse
motivo, faz-se necessário um levantamento profundo e detalhado das pesquisas e
desenvolvimento referentes as abstrações de processos. Acreditamos que um
\textit{survey} com tal temática traria novas perspectivas de pesquisa para a
área de SO. Além deste trabalho, o Capítulo \ref{cap:validacoes} apontou a
falta de um trabalho que gere uma taxonomia das aplicações de forma similar ao
que é feito na biologia.  Tal taxonomia pode ser útil para generalizar certas
características das aplicações e assim facilitar a seleção de subgrupos para
serem utilizadas em outros trabalho (p.ex., o Capítulo \ref{cap:validacoes}
poderia ter feito uso de tal taxonomia);

O Capítulo \ref{cap:validacoes} em conjunto com a experiência adquirida sobre o
Kernel Linux e os experimentos realizados (Capítulo \ref{cap:estudo-de-caso}),
revelaram a necessidade de uma ferramenta automatizada para testar falhas de
segurança. Seria interessante construir uma ferramenta que de forma sistemática
tente explorar as falhas de segurança já catalogadas e assim verificar se
alguma regressão aconteceu. Por exemplo, o openSSL apresentou a vulnerabilidade
\textit{Heartbleed} nas versões 1.0.1 e que foram corrigidas, nesse contexto
seria interessante ter um teste que verifique se tal ataque ainda é possível.
Note que tal ferramenta seria extremamente valiosa na demonstração da utilidade
de novas abstrações de processos. Uma determinada proposta poderia utilizar uma
versão comprometida de uma determinada aplicação junto com o teste que explora
uma falha e assim demonstrar que a sua proposta de extensão da abstração de
processos resolve algum problema de segurança.

A Seção \ref{sec:micro} abordou o tema dos \textit{microbenchmarks} tendo como
foco as validações para novas abstrações de processos. Nesse contexto, seria
interessante a construção de uma ferramenta de \textit{microbenchmark}
automatizada que possa ser utilizada para validar as novas abstrações. Tal
ferramenta tem que ser simples de ser alterada para que seja possível validar
as diferentes propostas sem grandes esforços de implementação.

Por fim, o trabalho futuro mais interessante consiste em implementar a nova
geração de abstrações de processos apresentada no Capítulo
\ref{cap:analise-sobre-abstracoes-de-processos} no GNU/Linux. Tal implementação
em um Kernel monolítico pode significar um avanço de como os SO modernos são
organizados e abrir caminhos para uma nova geração de SOs. Além disso, novos
paradigmas, técnicas e padrões de projetos podem torna-se possíveis com a nova
geração de abstrações de processos.

\section{Outras Contribuições do Mestrado}

Durante esse mestrado, busquei investir na minha formação passando pelas
seguintes áreas: ensino, pesquisa, orientação e desenvolvimento. Nesse sentido,
segue o resumo de outras contribuições que tive durante o mestrado:

\begin{enumerate}
  \item Publicações como primeiro autor:
    \begin{itemize}
      \item Continuous delivery: building trust in a large-scale, complex government organization - IEEE Software
      \item The Next-Generation OS Process Abstraction - IEEE Computer (Aguardando revisão)
    \end{itemize}

  \item Publicações como co-autor:
    \begin{itemize}
      \item FLOSS Project Management in Government-Academia Collaboration
      \item Brazilian Public Software Portal: an integrated platform for collaborative development
      \item Optimizing a Boundary Elements Method for Stationary Elastodynamic Problems implementation with GPUs (Prêmio de segundo melhor paper)
      \item Leading successful government-academia collaboration using FLOSS and agile values (Aguardando Revisão)
    \end{itemize}
  \item Orientandos:
    \begin{itemize}
      \item
Giuliano A. F. Belinassi: Indiretamente orientei o Giuliano no TCC, e na
produção do artigo: Optimizing a Boundary Elements Method for Stationary
Elastodynamic Problems implementation with GPUs (Prêmio de segundo melhor
paper). Durante o mestrado dele, eu tenho auxiliado na definição do tema.

    \end{itemize}
  \item Contribuições para Software livre:
    \begin{itemize}
      \item Kernel Linux:
        \begin{itemize}
          \item VKMS: Fui um dos autores do driver VKMS
          \item IIO: Ajudei a fundar o grupo de estudo de Kernel na USP. Esse grupo já colaborou movendo mais de três drivers da staging do Linux para a árvore princípal. Veja: \url{flusp.ime.usp.br}.
          \item Tenho mais de 30 patches na árvore princípal do Kernel
        \end{itemize}
      \item Debian:
        \begin{itemize}
          \item Mantenedor de alguns pacotes, em especial do Dia
          \item Promovi diversos eventos sobre o Debian em São Paulo e Brasília
        \end{itemize}
    \end{itemize}

  \item Apresentações:
    \begin{itemize}
      \item Linuxdev-br: Da linuxdev-br ao GSoC: Displays Virtuais no Kernel (\url{https://youtu.be/S0hBHiiTDjA})
      \item XDC19: Da linuxdev-br ao GSoC: Displays Virtuais no Kernel (\url{https://youtu.be/ber_9vkj_-U})
      \item The 14th International Conference on Open Source Systems: FLOSS Project Management in Government-Academia Collaboration
    \end{itemize}

  \item Outros:
    \begin{itemize}
      \item Participei do programa Google Summer of Code 2018
    \end{itemize}
\end{enumerate}
