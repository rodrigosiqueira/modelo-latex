%% ------------------------------------------------------------------------- %%
\imgchapter[right]{3cm}{chapter_7_title}{Conclusões}
\label{cap:conclusoes}

Ao longo deste trabalho investigamos diversos aspectos referentes às abstrações
de processos. Começamos revisitando conceitos já consolidados (Capítulo
\ref{cap:fundamentacao}) para que em seguida pudéssemos expandir a visão sobre
os processos (procuramos apresentar diferentes perspectivas sobre esse conceito).
No Capítulo~\ref{cap:trabalhos-analisados} elencamos e resumimos diversos
trabalhos relacionados as abstrações de processos fornecendo assim um panorama
geral sobre as pesquisas já realizadas. Buscamos o estado da arte e suas
intersecções com o estado da prática utilizando a implementação do MVAS como
estudo de caso (Capítulo \ref{cap:estudo-de-caso}). Além disso, durante este
trabalho, interagimos com a indústria por meio do Kernel Linux e com uma
parceria com a HPe (tal parceria durou apenas alguns meses). Por várias vezes
tentamos consertar o MVAS, juntamente com alguns funcionários da HPe, não
atingindo exito em tal tarefa; contudo, extraímos dessa experiência diversas
ideias nas quais refinamos no decorrer deste mestrado.

Durante o aprofundamento na pesquisa sobre novas abstrações de processos
percebemos que, por vários anos, a academia vem buscando inovações nesse campo,
mas tais progressos não chegam aos SOs de uso cotidiano. Como o Capítulo
\ref{cap:validacoes} discute, parte disso vem do fato de que algumas propostas
negligenciam aspectos importantes que precisam ser considerados para avançar
sistemas usados em produção. Por exemplo, é fundamental que as novas gerações
de abstração de processos levem em consideração aplicações amplamente adotadas.
No Capítulo \ref{cap:validacoes}, evidenciamos um subconjunto de aplicações que
podem servir como alicerce para demonstrar o uso e potencial de adoção de uma
nova abstração. Além disso, não basta levar ganhos em uma área ao custo de
fazer com que outras partes da aplicação sejam afetadas negativamente, é preciso
fazer uma análise geral dos impactos gerados por novas abstrações. Como
exemplo, o estudo de caso apresentado no Capítulo~\ref{cap:estudo-de-caso}
revelou como uma determinada proposta pode comprometer todo o sistema. Por
fim, muitos dos trabalhos analisados não realizam testes de estresse
consistentes, o que não revela a real natureza dos impactos.

Por outro lado, notamos da análise dos diversos trabalhos que não há uma
tentativa de unificar as propostas, normalmente cada trabalho busca
apresentar-se como diferentes de todos os outros já feitos. Após o estudo das
diversas propostas, notamos que a melhor forma de levar os avanços sugeridos
pelos pesquisadores para os SOs de uso cotidiano é buscando unificar as
pesquisas e adaptando elas para as necessidades atuais. No Capítulo
\ref{cap:analise-sobre-abstracoes-de-processos} buscamos mostrar tal aspecto
apontando um caminho para a próxima geração de abstrações de processos.
Qualquer tentativa de avançar o estado da prática também deve buscar o menor
impacto possível nos SOs.

Os anos 80 e 90 representaram o período áureo da pesquisa em SO; parte disso
deve-se ao fato de que tal área era muito jovem e carente de novas
definições. Com o natural amadurecimento da área tornou-se cada vez mais
complicado e menos atraente realizar pesquisa em SOs; pode-se dizer que hoje, o
interesse por tal área é relativamente morno quando comparado com outras áreas.
Parte desse momento dormente da pesquisa em SO pode ser fundamento no
distanciamento entre indústria e academia ou ainda na falta de novos
paradigmas em SO. Este trabalho é uma modesta tentativa de discutir
novos caminhos para a pesquisa em SO com a esperança de contribuir de forma a
inspirar novos e velhos pesquisadores a olhar essa área com mais interesse.

\section{Trabalhos Futuros}

% <chaws> acho que um ponto de partida seria pegar as CVEs existentes de kernel e tentar prioriza-las pra iniciar os testes

Este trabalho teve como objetivo estudar e colocar em evidência um campo de
pesquisa com pouca visibilidade. Dentre os trabalhos futuros que identificamos,
nós destacamos: escrever um \textit{survey} sobre o estado da arte das
abstrações de processos, a criação de um sistema de validações de abstrações de
processos que funcione de forma automatizada e a implementação da abstração de
processos da próxima geração proposta no
Capítulo~\ref{cap:analise-sobre-abstracoes-de-processos}.

Como podemos observar no Capítulo \ref{cap:trabalhos-analisados}, diversos
pesquisadores vem colocando esforços para fazer com que as abstrações de
processo evoluam. Além disso, as abstrações de processos adotadas na prática
também distinguem-se daquelas apresentadas na bibliografia padrão. Por esse
motivo, faz-se necessário um levantamento profundo e detalhado das pesquisas
referentes às abstrações de processos. Acreditamos que um
\textit{survey} com tal temática traria novas perspectivas de pesquisa para a
área de SO. Além desse trabalho, o Capítulo \ref{cap:validacoes} apontou a
falta de um trabalho que apresente uma taxonomia das aplicações de forma
similar ao que é feito na biologia.  Tal taxonomia pode ser útil para
generalizar certas características das aplicações e assim facilitar a seleção
de subgrupos de software para serem utilizadas em outros trabalho (p.ex., o
Capítulo \ref{cap:validacoes} poderia ter feito uso de tal taxonomia).

O assunto tratado no Capítulo \ref{cap:validacoes} em conjunto com a experiência adquirida sobre o
núcleo Linux e os experimentos realizados (Capítulo \ref{cap:estudo-de-caso}),
revelaram a necessidade de uma ferramenta automatizada para testar falhas de
segurança. Seria interessante construir um software que de forma sistemática
tente explorar as falhas de segurança já catalogadas e assim verificar se
alguma regressão aconteceu. Por exemplo, o openSSL apresentou a vulnerabilidade
\textit{Heartbleed} nas versões 1.0.1 e que foram corrigidas. Nesse contexto
seria interessante ter um teste que verifique se tal ataque ainda é possível.
Note que tal ferramenta seria extremamente valiosa na demonstração da utilidade
de novas abstrações de processos. Uma determinada proposta poderia utilizar uma
versão comprometida de uma determinada aplicação junto com o teste que explora
uma falha e assim demonstrar que a sua proposta de extensão da abstração de
processos resolve algum problema de segurança.

A Seção \ref{sec:micro} abordou o tema dos \textit{microbenchmarks} tendo como
foco as validações para novas abstrações de processos. Nesse contexto, seria
interessante a construção de um arcabouço de \textit{microbenchmark}
automatizado que possa ser utilizado para validar as novas abstrações. Tal
ferramenta tem que ser simples de ser alterada para que seja possível validar
as diferentes propostas sem grandes esforços de implementação.

Por fim, o trabalho futuro mais interessante consiste em implementar a nova
geração de abstrações de processos apresentada no Capítulo
\ref{cap:analise-sobre-abstracoes-de-processos}. Tal implementação
em um Kernel monolítico pode significar um avanço de como os SO modernos são
organizados e abrir caminhos para uma nova geração de SOs. Além disso, novos
paradigmas, técnicas e padrões de projetos podem emergir com a nova
geração de abstrações de processos.

\section{Outras Contribuições deste Mestrado}

Durante este mestrado, busquei investir na minha formação passando pelas
seguintes áreas: ensino, pesquisa, orientação e desenvolvimento. Nesse sentido,
segue o resumo de outras contribuições que tive durante o mestrado:

\begin{enumerate}
  \item Publicações como primeiro autor:
    \begin{itemize}
      \item
SIQUEIRA, Rodrigo et al. Continuous delivery: building trust in a large-scale, complex government organization. IEEE Software, v. 35, n. 2, p. 38-43, 2018.
      \item \emph{The Next-Generation OS Process Abstraction} -- IEEE Computer -- (Aguardando para submeter)
    \end{itemize}

  \item Publicações como co-autor:
    \begin{itemize}
			\item
WEN, Melissa et al. FLOSS Project Management in Government-Academia Collaboration. In: IFIP International Conference on Open Source Systems. Springer, Cham, 2018. p. 15-25. -- (Prêmio de melhor artigo)
      \item 
MEIRELLES, Paulo et al. Brazilian Public Software Portal: an integrated platform for collaborative development. In: Proceedings of the 13th International Symposium on Open Collaboration. ACM, 2017. p. 16.
      \item
BELINASSI, Giuliano AF et al. Optimizing a Boundary Elements Method for Stationary Elastodynamic Problems implementation with GPUs. WSCAD-WIC 2017, p. 51. -- (Prêmio de segundo melhor artigo da WSCAD-WIC)
      \item
WEN, Melissa et al. Leading successful government-academia collaboration using FLOSS and agile values. In: Journal of Systems and Software -- (Aguardando Revisão)
    \end{itemize}
  \item Trabalhos relacionados a orientandos:
    \begin{itemize}
      \item
Giuliano A. F. Belinassi: Auxiliei na orientação do Giuliano no seu TCC, e na
produção do artigo: \emph{Optimizing a Boundary Elements Method for Stationary
Elastodynamic Problems implementation with GPUs} (Prêmio de segundo melhor
artigo na WSCAD-WIC). Durante o mestrado dele, eu tenho auxiliado na definição
do tema e na construção de experimentos.
    \end{itemize}
  \item Contribuições para Software livre:
    \begin{itemize}
      \item Kernel Linux:
        \begin{itemize}
          \item VKMS: Fui um dos autores do driver VKMS
          \item IIO: Ajudei a fundar o grupo de estudo de Kernel na USP. Esse grupo já colaborou movendo mais de três \emph{drivers} da \emph{staging} do Linux para a árvore princípal. Veja: \url{https://flusp.ime.usp.br}.
          \item Tenho mais de 30 contribuições de código no repositório principal do Linux
        \end{itemize}
      \item Debian:
        \begin{itemize}
          \item Mantenedor de alguns pacotes, em especial do Dia
          \item Promovi diversos eventos sobre o Debian em São Paulo e em Brasília
        \end{itemize}
    \end{itemize}

  \item Apresentações:
    \begin{itemize}
      \item Linuxdev-br: Da linuxdev-br ao GSoC: Displays Virtuais no Kernel (\url{https://youtu.be/S0hBHiiTDjA})
      \item XDC18: VKMS (\url{https://youtu.be/ber_9vkj_-U})
      \item The 14th International Conference on Open Source Systems: FLOSS Project Management in Government-Academia Collaboration
    \end{itemize}

  \item Outros:
    \begin{itemize}
      \item
Participei do programa \emph{Google Summer of Code 2018} sob o guarda-chuva da
\emph{Xorg Foundation}. Trabalhei em conjunto com uma desenvolvedora para criar
as bases de um novo driver chamado VKMS (para mais detalhes, veja
\url{https://siqueira.tech/report/gsoc-final-report/})

    \end{itemize}
\end{enumerate}
