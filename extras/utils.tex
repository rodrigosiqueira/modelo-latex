%%%%%%%%%%%%%%%%%%%%%%%%%%%%%%%%%%%%%%%%%%%%%%%%%%%%%%%%%%%%%%%%%%%%%%%%%%%%%%%%
%%%%%%%%%%%%%%%%%%%%%%%%%%%%% ÍNDICE E HIPERLINKS %%%%%%%%%%%%%%%%%%%%%%%%%%%%%%
%%%%%%%%%%%%%%%%%%%%%%%%%%%%%%%%%%%%%%%%%%%%%%%%%%%%%%%%%%%%%%%%%%%%%%%%%%%%%%%%

% Cria índice remissivo. Este pacote precisa ser carregado antes de hyperref.
% A criação do índice remissivo depende de um programa auxiliar, que pode ser
% o "makeindex" (default) ou o xindy. xindy é mais poderoso e lida melhor com
% línguas diferentes e caracteres acentuados, mas o programa não está mais
% sendo mantido e índices criados com xindy não funcionam em conjunto com
% hyperref. Se quiser utilizar xindy mesmo assim, é possível contornar esse
% segundo problema configurando hyperref para *não* gerar hyperlinks no
% índice (mais abaixo) e configurando xindy para que ele gere esses hyperlinks
% por conta própria. Para isso, modifique a chamada ao pacote imakeidx (aqui)
% e altere as opções do pacote hyperref.
\providecommand\theindex{} % evita erros de compilação se a classe não tem index
%\usepackage[xindy]{imakeidx} % usando xindy
\usepackage{imakeidx} % usando makeindex

% Cria o arquivo de configuração para xindy lidar corretamente com hyperlinks.
\begin{filecontents*}{hyperxindy.xdy}
(define-attributes ("emph"))
(markup-locref :open "\hyperpage{" :close "}" :attr "default")
(markup-locref :open "\textbf{\hyperpage{" :close "}}" :attr "textbf")
(markup-locref :open "\textit{\hyperpage{" :close "}}" :attr "textit")
(markup-locref :open "\emph{\hyperpage{" :close "}}" :attr "emph")
\end{filecontents*}

% Cria o arquivo de configuração para makeindex colocar um cabeçalho
% para cada letra do índice.
\begin{filecontents*}{mkidxhead.ist}
headings_flag 1
heading_prefix "{\\bfseries "
heading_suffix "}\\nopagebreak\n"
\end{filecontents*}

% Por padrão, o cabeçalho das páginas do índice é feito em maiúsculas;
% vamos mudar isso e deixar fancyhdr definir a formatação
\indexsetup{
  othercode={\chaptermark{\indexname}},
}

\makeindex[
  noautomatic,
  intoc,
  % Estas opções são usadas por xindy
  % "-C utf8" ou "-M lang/latin/utf8.xdy" são truques para contornar este
  % bug, que existe em outras distribuições tambem:
  % https://bugs.launchpad.net/ubuntu/+source/xindy/+bug/1735439
  % Se "-C utf8" não funcionar, tente "-M lang/latin/utf8.xdy"
  %options=-C utf8 -M hyperxindy.xdy,
  %options=-M lang/latin/utf8.xdy -M hyperxindy.xdy,
  % Estas opções são usadas por makeindex
  options=-s mkidxhead.ist -l -c,
]

% O comando \ref por padrão mostra apenas o número do elemento a que se
% refere; assim, é preciso escrever "veja a Figura \ref{grafico}" ou
% "como visto na Seção \ref{sec:introducao}". Usando o pacote hyperref
% (carregado mais abaixo), esse número é transformado em um hiperlink.
%
% Se você quiser mudar esse comportamento, ative as packages varioref
% e cleveref e também as linhas "labelformat" e "crefname" mais abaixo.
% Nesse caso, você deve escrever apenas "veja a \ref{grafico}" ou
% "como visto na \ref{sec:introducao}" etc. e o nome do elemento será
% gerado automaticamente como hiperlink.
%
% Se, além dessa mudança, você quiser usar os recursos de varioref ou
% cleveref, mantenha as linhas labelformat comentadas e use os comandos
% \vref ou \cref, conforme sua preferência, também sem indicar o nome do
% elemento, que é inserido automaticamente. Vale lembrar que o comando
% \vref de varioref pode causar problemas com hyperref, impedindo a
% geração do PDF final.
%
% ATENÇÃO: varioref, hyperref e cleveref devem ser carregadas nessa ordem!
%\usepackage{varioref}

%\labelformat{figure}{Figura~#1}
%\labelformat{table}{Tabela~#1}
%\labelformat{equation}{Equação~#1}
%% Isto não funciona corretamente com os apêndices; o comando seguinte
%% contorna esse problema
%%\labelformat{chapter}{Capítulo~#1}
%\makeatletter
%\labelformat{chapter}{\@chapapp~#1}
%\makeatother
%\labelformat{section}{Seção~#1}
%\labelformat{subsection}{Seção~#1}
%\labelformat{subsubsection}{Seção~#1}

% Usamos "PassOptions" aqui porque outras packages definem opções para
% hyperref também e chamar a package com opções diretamente gera conflitos.
\PassOptionsToPackage{
  unicode=true,
  plainpages=false,
  pdfpagelabels,
  colorlinks=true,
  %citecolor=black,
  %linkcolor=black,
  %urlcolor=black,
  %filecolor=black,
  citecolor=DarkGreen,
  linkcolor=NavyBlue,
  urlcolor=DarkRed,
  filecolor=green,
  bookmarksopen=true,
  % hyperref não gera hyperlinks corretos em índices remissivos criados com
  % xindy; assim, é possível desabilitar essa função aqui e gerar os
  % hyperlinks através da configuração de xindy definida anteriormente. Com
  % makeindex (o default), quem precisa criar os hyperlinks é hyperref.
  %hyperindex=false,
}{hyperref}

% Cria hiperlinks para capítulos, seções, \ref's, URLs etc.
\usepackage{hyperref}

%\usepackage[nameinlink,noabbrev,capitalise]{cleveref}
%% cleveref não tem tradução para o português
%\crefname{figure}{Figura}{Figuras}
%\crefname{table}{Tabela}{Tabelas}
%\crefname{chapter}{Capítulo}{Capítulos}
%\crefname{section}{Seção}{Seções}
%\crefname{subsection}{Seção}{Seções}
%\crefname{subsubsection}{Seção}{Seções}
%\crefname{appendix}{Apêndice}{Apêndices}
%\crefname{subappendix}{Apêndice}{Apêndices}
%\crefname{subsubappendix}{Apêndice}{Apêndices}

% ao criar uma referência hyperref para um float, a referência aponta
% para o final do caption do float, o que não é muito bom. Este pacote
% faz a referência apontar para o início do float (é possível personalizar
% também). Esta package é incompatível com a classe beamer (usada para
% criar posters e apresentações), então testamos a compatibilidade antes
% de carregá-la.
\ifboolexpr{
  test {\ifcsdef{figure}} and
  test {\ifcsdef{figure*}} and
  test {\ifcsdef{table}} and
  test {\ifcsdef{table*}}
}{\usepackage[all]{hypcap}}{}

% XMP (eXtensible Metadata Platform) is a mechanism proposed by Adobe for
% embedding document metadata within the document itself. The package
% integrates seamlessly with hyperref and requires virtually no modifications
% to documents that already exploit hyperref's mechanisms for specifying PDF
% metadata.
\usepackage{hyperxmp}

% A classe Book inclui o comando \appendix, que (obviamente) permite inserir
% apêndices no documento. No entanto, não há suporte similar para anexos. Esta
% package (que não é padrão do LaTeX, foi criada para este modelo) define o
% comando \annex. Ela deve ser carregada depois de hyperref.
\usepackagefromsubdir{extras/}{annex}


%%%%%%%%%%%%%%%%%%%%%%%%%%%%%%%%%%%%%%%%%%%%%%%%%%%%%%%%%%%%%%%%%%%%%%%%%%%%%%%%
%%%%%%%%%%%%%%%%%%%%%%%%%%%% OUTROS PACOTES ÚTEIS %%%%%%%%%%%%%%%%%%%%%%%%%%%%%%
%%%%%%%%%%%%%%%%%%%%%%%%%%%%%%%%%%%%%%%%%%%%%%%%%%%%%%%%%%%%%%%%%%%%%%%%%%%%%%%%

% Você provavelmente vai querer ler a documentação de alguns destes pacotes
% para personalizar algum aspecto do trabalho ou usar algum recurso específico.

% Formatação personalizada das listas "itemize", "enumerate" e
% "description", além de permitir criar novos tipos de listas
%\usepackage{paralist}

% Trechos de texto "puro" (tabs, quebras de linha etc. não são modificados)
\usepackage{verbatim}

% Sublinhado e outras formas de realce de texto
\usepackage{soul}
\usepackage{soulutf8}

% Recursos e símbolos adicionais para o modo matemático
% para evitar problemas de compatibilidade com algumas fontes, o pacote
% amsthm já foi carregado mais acima
\usepackage{latexsym}
\usepackage{amsmath}
\usepackage{amssymb}
\usepackage{mathtools}
%\usepackage{MnSymbol}
%\usepackage{stmaryrd}

% Notação bra-ket
%\usepackage{braket}

%\num \SI and \SIrange. For example, \SI{10}{\hertz} \SIrange{10}{100}{\hertz}
%\usepackage[binary-units]{siunitx}

% Citações melhores; se você pretende fazer citações de textos
% relativamente extensos, vale a pena ler a documentação. biblatex
% utiliza recursos deste pacote.
\usepackage{csquotes}

\usepackage{url}
% URL com fonte sem serifa ao invés de teletype
\urlstyle{sf}

% Permite inserir comentários, muito bom durante a escrita do texto
\usepackage[textsize=scriptsize,colorinlistoftodos,textwidth=2.5cm]{todonotes}
\presetkeys{todonotes}{color=orange!40!white}{}

% Comando para fazer notas com highlight no texto correspondente:
% \hltodo[texto][opções]{comentário}
\makeatletter
\if@todonotes@disabled
  \NewDocumentCommand{\hltodo}{O{} O{} +m}{#1}
\else
  \NewDocumentCommand{\hltodo}{O{} O{} +m}{
    \ifstrempty{#1}{}{\texthl{#1}}%
    \todo[#2]{#3}{}%
  }
\fi
\makeatother

% para formatar código-fonte (ex. em Java).
\usepackage{listings}

% O pacote listings não lida bem com acentos! No caso dos caracteres acentuados
% usados em português é possível contornar o problema com a tabela abaixo.
% From https://en.wikibooks.org/wiki/LaTeX/Source_Code_Listings#Encoding_issue
\lstset{literate=
  {á}{{\'a}}1 {é}{{\'e}}1 {í}{{\'i}}1 {ó}{{\'o}}1 {ú}{{\'u}}1
  {Á}{{\'A}}1 {É}{{\'E}}1 {Í}{{\'I}}1 {Ó}{{\'O}}1 {Ú}{{\'U}}1
  {à}{{\`a}}1 {è}{{\`e}}1 {ì}{{\`i}}1 {ò}{{\`o}}1 {ù}{{\`u}}1
  {À}{{\`A}}1 {È}{{\'E}}1 {Ì}{{\`I}}1 {Ò}{{\`O}}1 {Ù}{{\`U}}1
  {ä}{{\"a}}1 {ë}{{\"e}}1 {ï}{{\"i}}1 {ö}{{\"o}}1 {ü}{{\"u}}1
  {Ä}{{\"A}}1 {Ë}{{\"E}}1 {Ï}{{\"I}}1 {Ö}{{\"O}}1 {Ü}{{\"U}}1
  {â}{{\^a}}1 {ê}{{\^e}}1 {î}{{\^i}}1 {ô}{{\^o}}1 {û}{{\^u}}1
  {Â}{{\^A}}1 {Ê}{{\^E}}1 {Î}{{\^I}}1 {Ô}{{\^O}}1 {Û}{{\^U}}1
  {œ}{{\oe}}1 {Œ}{{\OE}}1 {æ}{{\ae}}1 {Æ}{{\AE}}1 {ß}{{\ss}}1
  {ű}{{\H{u}}}1 {Ű}{{\H{U}}}1 {ő}{{\H{o}}}1 {Ő}{{\H{O}}}1
  {ç}{{\c c}}1 {Ç}{{\c C}}1 {ø}{{\o}}1 {å}{{\r a}}1 {Å}{{\r A}}1
  {€}{{\euro}}1 {£}{{\pounds}}1 {«}{{\guillemotleft}}1
  {»}{{\guillemotright}}1 {ñ}{{\~n}}1 {Ñ}{{\~N}}1 {¿}{{?`}}1
}

% Opções default para o pacote listings
% Ref: http://en.wikibooks.org/wiki/LaTeX/Packages/Listings
\lstset{
  basicstyle=\footnotesize\ttfamily,  % the font that is used for the code
  numbers=left,                       % where to put the line-numbers
  numberstyle=\footnotesize\ttfamily, % the font that is used for the line-numbers
  stepnumber=1,                       % the step between two line-numbers. If it's 1 each line will be numbered
  numbersep=20pt,                     % how far the line-numbers are from the code
  commentstyle=\color{Brown}\upshape,
  stringstyle=\color{black},
  identifierstyle=\color{DarkBlue},
  keywordstyle=\color{cyan},
  showspaces=false,                   % show spaces adding particular underscores
  showstringspaces=false,             % underline spaces within strings
  showtabs=false,                     % show tabs within strings adding particular underscores
  frame=single,                       % adds a frame around the code
  framerule=0.6pt,
  tabsize=2,                          % sets default tabsize to 2 spaces
  captionpos=b,                       % sets the caption-position to bottom
  breaklines=true,                    % sets automatic line breaking
  breakatwhitespace=false,            % sets if automatic breaks should only happen at whitespace
  escapeinside={\%*}{*)},             % if you want to add a comment within your code
  backgroundcolor=\color[rgb]{1.0,1.0,1.0}, % choose the background color.
  rulecolor=\color{darkgray},
  extendedchars=true,
  inputencoding=utf8,
  xleftmargin=30pt,
  xrightmargin=10pt,
  framexleftmargin=25pt,
  framexrightmargin=5pt,
  framesep=5pt,
}

% Um exemplo de estilo personalizado para listings (tabulações maiores)
\lstdefinestyle{wider} {
  tabsize = 4,
  numbersep=15pt,
  xleftmargin=25pt,
  framexleftmargin=20pt,
}

% Outro exemplo de estilo personalizado para listings (sem cores)
\lstdefinestyle{nocolor} {
  commentstyle=\color{darkgray}\upshape,
  stringstyle=\color{black},
  identifierstyle=\color{black},
  keywordstyle=\color{black},
}

% Um exemplo de definição de linguagem para listings (XML)
\lstdefinelanguage{XML}{
  morecomment=[s]{<!--}{-->},
  morecomment=[s]{<!-- }{ -->},
  morecomment=[n]{<!--}{-->},
  morecomment=[n]{<!-- }{ -->},
  morestring=[b]",
  morestring=[s]{>}{<},
  morecomment=[s]{<?}{?>},
  morekeywords={xmlns,version,type}% list your attributes here
}

% Símbolos adicionais, como \celsius, \ohm, \perthousand etc.
%\usepackage{gensymb}

% Símbolos adicionais, como \textrightarrow, \texteuro etc.
\usepackage{textcomp}

% Permite criar listas como glossários, listas de abreviaturas etc.
% https://en.wikibooks.org/wiki/LaTeX/Glossary
%\usepackage{glossaries}

% Permite formatar texto em colunas
\usepackage{multicol}

% Gantt charts; útil para fazer o cronograma para o exame de
% qualificação, por exemplo.
\usepackage{pgfgantt}

% Na versão 5 do pacote pgfgantt, a opção "compress calendar"
% deixou de existir, sendo substituída por "time slot unit=month".
% Aqui, um truque para a configuração abaixo funcionar com ambas
% as versões.
\makeatletter
\newcommand*\columnsAsMonths{%
  \@ifpackagelater{pgfgantt}{2018/01/01}{time slot unit=month}{compress calendar}%
}
\makeatother

% Estes parâmetros definem a aparência das gantt charts e variam
% em função da fonte do documento.
\ganttset{
    time slot format=isodate-yearmonth,
    %time slot unit=month, % versão 5 ou posterior
    %compress calendar, % versão anterior à 5
    \columnsAsMonths, % veja mais abaixo
    vgrid,
    x unit=1.7em,
    y unit title=3ex,
    y unit chart=4ex,
    % O "strut" é necessário para alinhar o baseline dos nomes dos meses
    title label font=\strut\footnotesize,
    group label font=\footnotesize\bfseries,
    bar label font=\footnotesize,
    milestone label font=\footnotesize\itshape,
    % "align=right" é necessário para \ganttalignnewline funcionar
    group label node/.append style={align=right},
    bar label node/.append style={align=right},
    milestone label node/.append style={align=right},
    group incomplete/.append style={fill=black!50},
    bar/.append style={fill=black!25,draw=black},
    bar incomplete/.append style={fill=white,draw=black},
    % Não é preciso imprimir "0%"
    progress label text=\ifnumequal{#1}{0}{}{(#1\%)},
    % Formato e tamanho dos elementos
    title height=.9,
    group top shift=.4,
    group left shift=0,
    group right shift=0,
    group peaks tip position=0,
    group peaks width=.2,
    group peaks height=.3,
    milestone height=.4,
    milestone top shift=.4,
    milestone left shift=.8,
    milestone right shift=.2,
}

% Em inglês, tanto o nome completo quanto a abreviação do mês de maio
% são "May"; por conta disso, na tradução em português LaTeX erra a
% abreviação. Como talvez usemos o nome inteiro do mês em outro lugar,
% ao invés de forçar a tradução para "Mai" globalmente, fazemos isso
% apenas em ganttchart.
\AtBeginEnvironment{ganttchart}{\deftranslation[to=Portuguese]{May}{Mai}}

% Os comandos \TeX e \LaTeX são nativos do LaTeX; esta package acrescenta os
% comandos \XeLaTeX e \LuaLaTeX. Você provavelmente não precisa desse recurso
% e, portanto, pode removê-la.
\usepackage{metalogo}
