\chapter{Definição de Software Livre}
\label{ann:softlivre}

Um programa é software livre se sua licença garante que os seus usuários possuem as seguintes quatro liberdades essenciais\footnote{\url{www.gnu.org/philosophy/free-sw.pt-br.html}}:
\begin{itemize}
  \item A liberdade de executar o programa como você desejar, para qualquer propósito (liberdade 0).
  \item A liberdade de estudar como o programa funciona, e adaptá-lo às suas necessidades (liberdade 1). Para tanto, acesso ao código-fonte é um pré-requisito.
  \item A liberdade de redistribuir cópias de modo que você possa ajudar outros (liberdade 2).
  \item A liberdade de distribuir cópias de suas versões modificadas a outros (liberdade 3). Desta forma, você pode dar a toda comunidade a chance de beneficiar de suas mudanças. Para tanto, acesso ao código-fonte é um pré-requisito.
\end{itemize}
