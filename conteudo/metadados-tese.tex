%%%%%%%%%%%%%%%%%%%%%%%%%%%%%%%%%%%%%%%%%%%%%%%%%%%%%%%%%%%%%%%%%%%%%%%%%%%%%%%%
%%%%%%%%%%%%%%%%%%%%%%%%%%%%% METADADOS DA TESE %%%%%%%%%%%%%%%%%%%%%%%%%%%%%%%%
%%%%%%%%%%%%%%%%%%%%%%%%%%%%%%%%%%%%%%%%%%%%%%%%%%%%%%%%%%%%%%%%%%%%%%%%%%%%%%%%

% Este pacote define o formato da capa, páginas de rosto, dedicatória e
% resumo. Se você pretende criar essas páginas manualmente, não precisa
% carregar este pacote nem definir os dados abaixo.
\dowithsubdir{extras/}{\usepackage{imeusp-capa}}

% Define o texto da capa e da referência que vai na página do resumo;
% "masc" ou "fem" definem se serão usadas palavras no masculino ou feminino
% (Mestre/Mestra, Doutor/Doutora, candidato/candidata).
\mestrado[masc]
%\doutorado[masc]

% Se "\title" está em inglês, você pode definir o título em português aqui
\tituloport{Uma Visão Sobre a Próxima Geração de Abstrações de Processos em Sistemas Operacionais}

% Se "\title" está em português, você pode definir o título em inglês aqui
\tituloeng{An Outline for the Next Generation of Process Abstractions in Operating Systems}

% Se o trabalho não tiver subtítulo, basta remover isto.
\subtitulo{}

% Se isto não for definido, "\subtitulo" é utilizado no lugar
\subtituloeng{}

\orientador[masc]{Prof. Dr. Fabio Kon}

% Se não houver, remova
% \coorientador[masc]{Prof. Dr. Ciclano}

% Se isto não for definido, "\coorientador" é utilizado no lugar
%\coorientadoreng{Prof. Dr. Ciclano}

\programa{Ciência da Computação}

% Se isto não for definido, "\programa" é utilizado no lugar
\programaeng{Computer Science}

% Se não houver, remova
\apoio{Durante o desenvolvimento deste trabalho o autor recebeu auxílio
financeiro da CAPES}

% Se isto não for definido, "\apoio" é utilizado no lugar
\apoioeng{During this work, the author was supported by XXX}

\localdefesa{São Paulo}

\datadefesa{6 de fevereiro de 2019}

% Se isto não for definido, "\datadefesa" é utilizado no lugar
\datadefesaeng{August 10th, 2017}

% Necessário para criar a referência do documento que aparece
% na página do resumo
\ano{2019}

\banca{
  \begin{itemize}
    \item Profª. Drª. Fabio Kon - IME-USP [sem ponto final]
    \item Prof. Dr. Nome Completo - IME-USP [sem ponto final]
    \item Prof. Dr. Nome Completo - IMPA [sem ponto final]
  \end{itemize}
}

% Se isto não for definido, "\banca" é utilizado no lugar
\bancaeng{
  \begin{itemize}
    \item Prof. Dr. Nome Completo (advisor) - IME-USP [sem ponto final]
    \item Prof. Dr. Nome Completo - IME-USP [sem ponto final]
    \item Prof. Dr. Nome Completo - IMPA [sem ponto final]
  \end{itemize}
}

% Palavras-chave separadas por ponto e finalizadas também com ponto.
\palavraschave{Abstração de processos. Gerenciamento de processos. Processos. Sistemas Operacionais. SO.}

\keywords{Processes abstractions. Processes management. Processes. Operating Systems. OS.}

% Se quiser estabelecer regras diferentes, converse com seu
% orientador
\direitos{Autorizo a reprodução e divulgação total ou parcial
deste trabalho, por qualquer meio convencional ou
eletrônico, para fins de estudo e pesquisa, desde que
citada a fonte.}

% Isto deve ser preparado em conjunto com o bibliotecário
%\fichacatalografica{
% nome do autor, título, etc.
%}
