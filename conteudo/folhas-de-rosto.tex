%%%%%%%%%%%%%%%%%%%%%%%%%%% CAPA E FOLHAS DE ROSTO %%%%%%%%%%%%%%%%%%%%%%%%%%%%%

% Embora as páginas iniciais *pareçam* não ter numeração, a numeração existe,
% só não é impressa. O comando \mainmatter (mais abaixo) reinicia a contagem
% de páginas e elas passam a ser impressas. Isso significa que existem duas
% páginas com o número "1": a capa e a página do primeiro capítulo. O pacote
% hyperref não lida bem com essa situação. Assim, vamos desabilitar hyperlinks
% para números de páginas no início do documento e reabilitar mais adiante.
\hypersetup{pageanchor=false}

% A capa; o parâmetro pode ser "port" ou "eng" para definir a língua
\capaime[port]
%\capaime[eng]

% Se você não quiser usar a capa padrão, você pode criar uma outra
% capa manualmente ou em um programa diferente. No segundo caso, é só
% importar a capa como uma página adicional usando o pacote pdfpages.
%\includepdf{./arquivo_da_capa.pdf}

% A página de rosto da versão para depósito (ou seja, a versão final
% antes da defesa) deve ser diferente da página de rosto da versão
% definitiva (ou seja, a versão final após a incorporação das sugestões
% da banca). Os parâmetros podem ser "port/eng" para a língua e
% "provisoria/definitiva" para o tipo de página de rosto.
%\pagrostoime[port]{definitiva}
\pagrostoime[port]{provisoria}
%\pagrostoime[eng]{definitiva}
%\pagrostoime[eng]{provisoria}

%%%%%%%%%%%%%%%%%%%% DEDICATÓRIA, RESUMO, AGRADECIMENTOS %%%%%%%%%%%%%%%%%%%%%%%

% A definição deste ambiente está no pacote imeusp; se você não
% carregar esse pacote, precisa cuidar desta página manualmente.
\begin{dedicatoria}
Dedico este trabalho para todo brasileiro que vem de camadas desprivilegiadas da sociedade, mas que com muito sacrifício nadam contra a maré de uma sociedade doente e conservadora. 
\end{dedicatoria}

% Após a capa e as páginas de rosto, começamos a numerar as páginas; com isso,
% podemos também reabilitar links para números de páginas no pacote hyperref.
% Isso porque, embora contagem de páginas aqui começe em 1 e no primeiro
% capítulo também, o fato de uma numeração usar algarismos romanos e a outra
% algarismos arábicos é suficiente para evitar problemas.
\pagenumbering{roman}
\hypersetup{pageanchor=true}

% Agradecimentos:
% Se o candidato não quer fazer agradecimentos, deve simplesmente eliminar
% esta página. A epígrafe, obviamente, é opcional; é possível colocar
% epígrafes em todos os capítulos. O comando "\chapter*" faz esta seção
% não ser incluída no sumário.

\chapter*{Agradecimentos}
\epigrafe{Se eu vi mais longe, foi por estar sobre ombros de gigantes.}{Isaac Newton}

Não existe outra forma de começar estes agradecimentos sem iniciar reconhecendo
o grupo de pessoas que mais me apoiaram e apoiam até hoje a minha família. Se
fiz coisas boas nesta vida, se fiz algo útil para a alguém, ou qualquer outra
coisa digna de reconhecimento, devo tudo isso a duas mulheres extraordinárias
que me inspiraram e inspiram até hoje; muito obrigado mãe (Simone Martins
Siqueira) e avó/mãe (Maria Francelina Martins Siqueira), duas mães que com
muito carinho, amor, paciência e ternura me moldaram. Sei que vocês passaram
por dificuldades enormes nesta vida, tiveram que lutar firmemente contra todos
aqueles que torciam contra, foram éticas e consistentes com os seus valores em
todas as situações. Muito obrigado por tudo o que vocês fizeram por mim, por
todas as batalhas travadas, em prol do meu enriquecimento intelectual. Agradeço
a minha super irmã (Sophia Fiama), que tenho como uma filha, que me enche de
orgulho e me inspira a sempre buscar o impossível. Por fim, agradeço muito a
minha esposa (Jozzi Quezzia) que é sempre tão carinhosa, amável, paciente e
companheira. Muito obrigado por ter ficado ao meu lado em tantos momentos de
luta e sacrifício.Você é uma das poucas pessoas que sabem de fato o que foram
esses últimos três anos de intensa luta. Sei que não foi fácil para você esses
últimos anos mas quero que saiba que sou realmente muito grato em te ter na
minha vida. Enfim, muito obrigado a todas as mulheres da minha vida que tanto
me auxiliaram e incentivaram.

Agradeço também a minha “nova velha família” na qual a minha esposa me deu a
honra de fazer parte ao se casar comigo. Obrigado seu José Eliziario e Dona
Maria Aparecida por sempre me tratar com tanto carinho, pelas brincadeiras e
por sempre me fazer esquecer um pouco dos problemas. Obrigado Bertineide por me
tratar bem e por cuidar da minha esposa até hoje. Muito obrigado a todos vocês
que vieram somar coisas boas na minha vida.

Devo um grande agradecimento ao meu amigo Paulo Meirelles e toda a sua família
por sempre tratarem eu e a minha esposa com tanto apreço. Obrigado Paulo por
ter me ensinado como construir um projeto de software de verdade, por ter me
ajudado a melhorar o meu “RH” e por me apresentar ao software livre; em
especial agradeço por ter sido um grande exemplo de ética e trabalho. Muito
obrigado Flavio e Patricia Kobayashi por todo suporte que vocês deram para mim
e toda a minha família durante esses anos; admiro muito vocês. Também agradeço
muito ao meu grande amigo Charles por sempre me dar bons conselhos, por me
ajudar em momentos difíceis, por se preocupar comigo e por ser sempre o melhor
companheiro de doritos com Bohemia. Agradeço ao meu amigo Paulo Márcio
Rodrigues, que foi um grande companheiro durante a minha caminhada pelo
mestrado fazendo companhia para mim nos fins de semana de trabalho no
laboratório ou nas altas horas de trabalho.

Agradeço ao meu orientador Fabio Kon por ter aceitado o desafio de me orientar
em uma área tão complexa. Obrigado por ter me ajudado a concluir essa fase da
minha vida e por ter me dado o suporte necessário para que eu pudesse melhorar
as minhas habilidades. Agradeço também ao super Nelson por todas as discussões
que tivemos, por ter me ajudado com inúmeras discussões e revisões. Nelson,
muito obrigado por todos os aconselhamentos, pelos ensinamentos e pelos bons
momentos que você proporcionou no dia-a-dia de laboratório.

Obrigado Renan por todos os momentos que tivemos juntos durante o mestrado, por
ter sido um grande amigo e pelas boas confusões que compartilhamos pelas filas
dos bandejões da USP. Obrigado Arthur, Athos, Thallysman, Dylan, Mourão e
Kanashiro por todas as nossas conversas referentes a pesquisa, debates sobre
software livre e por toda “água” compartilhada. Agradeço também a Melissa Wen
por todas as suas revisões e pelo time vencedor que montamos juntos. Obrigado
Giuliano por sua amizade e companheirismo de lab; obrigado também por ter
confiado nas minhas orientações. Agradeço a Isabel pelas suas várias tentativas
desajeitadas e ligeiramente vagas, mas de coração, de me ajudar; apesar de
tudo, você também fez parte deste ciclo.

Agradeço a USP e UnB por todas as oportunidades, sinto-me privilegiado por ter
passado por essas instituições. Agradeço ao CNPq pelo investimento
proporcionado a minha formação ao longo desses anos, e prometo  que honrarei
meu país, com os conhecimentos adquiridos com excelência e eficiência.  Por
fim, agradeço ao Senhor Josenaldo Ferreira Batista pelo excelente trabalho que
vem desenvolvendo ao longo desses anos junto ao serviço de prestação de contas
do CNPq durante todos esses anos,  a pesquisa e os que fazem parte do corpo
docente e discente só tem a agradecer.

% O resumo é obrigatório, em português e inglês. Este comando também gera
% automaticamente a referência para o próprio documento, conforme as normas
% sugeridas da USP
\begin{resumo}{port}

Nas últimas décadas, muitos pesquisadores dedicaram-se a avançar o modelo atual
de abstração de processos, seja por meio da adição de camadas extras de
segurança, seja em busca de melhorias de desempenho, ou ainda com o objetivo de
fornecer suporte para novos recursos de hardware. Tais melhorias são relevantes
porque abstrações de processos em SOs de propósito geral representam o ponto de
encontro de diversos recursos de interesse dos usuários. Processos representam
a convergência entre a aplicação dos usuários, os modelos de programação
oferecidos pelo SO e a utilização dos recursos de hardware. Os esforços para
expandir as capacidades dos SOs no nível da abstração de processos abrem uma
nova área de pesquisa ainda pouco explorada. Nesta dissertação, após um
levantamento preliminar dos trabalhos relacionados ao tema, nos concentramos em
9 pesquisas que foram selecionadas levando-se em consideração aspectos como as
propostas de implementação adotadas por elas e o seu impacto na literatura da área.
Desses trabalhos, derivamos um conjunto de características que consideramos
importantes para guiar o desenvolvimento da próxima geração de abstrações de
processos. Partindo de tais características, propomos um modelo teórico chamado
de \emph{bead} cujo o objetivo é ilustrar os desafios e vantagem em se expandir
as abstrações de processos. Além disso, sugerimos uma coleção de
\emph{microbenchmarks} que podem ser utilizados para revelar parte dos impactos
de novas abstrações de processos. Por fim, realizamos uma discussão sobre
aplicações de uso cotidiano que podem ser utilizadas para a validação dessas
propostas e que também possam delas se beneficiar.

\end{resumo}

% O resumo é obrigatório, em português e inglês. Este comando também gera
% automaticamente a referência para o próprio documento, conforme as normas
% sugeridas da USP
\begin{resumo}{eng}

In recent decades, many researchers committed to pushing forward the current
model of process abstraction, either by adding extra layers of security or
seeking performance improvements or even providing support for new hardware.
Such enhancements are relevant because process abstractions in general-purpose
OSes represent the meeting point of several aspects of users concern. Processes
join together user applications, programming models provided by the OS and
hardware resources access. Efforts to expand OS capabilities at the process
abstraction level represent a new and underexplored research field. In this
thesis, after a preliminary survey of the area, we selected nine works to focus
on by considering aspects such as their implementation approach and their
impact on the literature. From these works, we extracted a set of
characteristics that we consider essential to guide the development of the next
generation of process abstractions. Based on such characteristics, we propose a
theoretical model called ``bead'', which illustrates the challenges and
advantages of expanding process abstractions. Furthermore, we suggest a
collection of microbenchmarks that can be used to reveal some of the impacts of
new process abstractions. Finally, we discuss real-world applications that can
be used to validate these proposals, and that could also benefit from them. 

\end{resumo}

% Como as listas que se seguem podem não incluir uma quebra de página
% obrigatória, inserimos uma quebra manualmente aqui.
\makeatletter
\if@openright\cleardoublepage\else\clearpage\fi
\makeatother

%%%%%%%%%%%%%%%%%%%%%%%%%%% LISTAS DE FIGURAS ETC. %%%%%%%%%%%%%%%%%%%%%%%%%%%%%

% Todas as listas são opcionais; Usando "\chapter*" elas não são incluídas
% no sumário. As listas geradas automaticamente também não são incluídas
% por conta das opções "notlot" e "notlof" que usamos mais acima.

% Normalmente, "\chapter*" faz o novo capítulo iniciar em uma nova página, e as
% listas geradas automaticamente também por padrão ficam em páginas separadas.
% Como cada uma destas listas é muito curta, não faz muito sentido fazer isso
% aqui, então usamos este comando para desabilitar essas quebras de página.
% Se você preferir, comente as linhas com esse comando e des-comente as linhas
% sem ele para criar as listas em páginas separadas. Observe que você também
% pode inserir quebras de página manualmente (com \clearpage, veja o exemplo
% mais abaixo).
\newcommand\disablenewpage[1]{{\let\clearpage\par\let\cleardoublepage\par #1}}

% Nestas listas, é melhor usar "raggedbottom" (veja basics.tex). Colocamos
% a opção correspondente e as listas dentro de um par de chaves para ativar
% raggedbottom apenas temporariamente.
{
\raggedbottom

%%%%% Listas criadas manualmente

%\chapter*{Lista de Abreviaturas}
\disablenewpage{\chapter*{Lista de Abreviaturas}}

\begin{tabular}{rl}
  MMP      & Mondrian Memory Protection\\
  PD       & Protection Domain\\
  SO       & Sistema Operacional\\
  SB       & Secure binding \\
  TLB      & Translation Lookaside Buffer \\
  VMM      & Virtual Machine Monitor \\
  MPM      & Multi-Processing Module \\
  APR      & Apache Portable Runtime \\
  GC       & Garbage Collection \\
  VMA      & Virtual Memory AREA\\
  PTE      & Page Table Entry\\
  USP      & Universidade de São Paulo
\end{tabular}

%\chapter*{Lista de Símbolos}
%{\let\cleardoublepage\relax \addvspace{55pt plus 15pt minus 15pt} \chapter*{Lista de Símbolos} }
%\begin{tabular}{rl}
%        $\omega$    & Frequência angular\\
%        $\psi$      & Função de análise \emph{wavelet}\\
%        $\Psi$      & Transformada de Fourier de $\psi$\\
%\end{tabular}

% Quebra de página manual
\clearpage

%%%%% Listas criadas automaticamente

%\listoffigures
\disablenewpage{\listoffigures}

%\listoftables
\disablenewpage{\listoftables}

% Esta lista é criada "automaticamente" pela package float quando
% definimos o novo tipo de float "program" (em utils.tex)
%\listof{program}{\proglistname}
\disablenewpage{\listof{program}{\proglistname}}

\disablenewpage{\listof{pseudocode}{\pseudocodelistname}}

} % Final de "raggedbottom"

% Sumário (obrigatório)
\tableofcontents

% Referências indiretas ("x", veja "y") para o índice remissivo (opcionais,
% pois o índice é opcional). É comum colocar esses itens no final do documento,
% junto com o comando \printindex, mas em alguns casos isso torna necessário
% executar texindy (ou makeindex) mais de uma vez, então colocar aqui é melhor.
%\index{Inglês|see{Língua estrangeira}}
%\index{Figuras|see{Floats}}
%\index{Tabelas|see{Floats}}
%\index{Código-fonte|see{Floats}}
%\index{Subcaptions|see{Subfiguras}}
%\index{Sublegendas|see{Subfiguras}}
%\index{Equações|see{Modo Matemático}}
%\index{Fórmulas|see{Modo Matemático}}
%\index{Rodapé, notas|see{Notas de rodapé}}
%\index{Captions|see{Legendas}}
%\index{Versão original|see{Tese/Dissertação, versões}}
%\index{Versão corrigida|see{Tese/Dissertação, versões}}
%\index{Palavras estrangeiras|see{Língua estrangeira}}
%\index{Floats!Algoritmo|see{Floats, Ordem}}
