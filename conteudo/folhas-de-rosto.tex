%%%%%%%%%%%%%%%%%%%%%%%%%%% CAPA E FOLHAS DE ROSTO %%%%%%%%%%%%%%%%%%%%%%%%%%%%%

% Embora as páginas iniciais *pareçam* não ter numeração, a numeração existe,
% só não é impressa. O comando \mainmatter (mais abaixo) reinicia a contagem
% de páginas e elas passam a ser impressas. Isso significa que existem duas
% páginas com o número "1": a capa e a página do primeiro capítulo. O pacote
% hyperref não lida bem com essa situação. Assim, vamos desabilitar hyperlinks
% para números de páginas no início do documento e reabilitar mais adiante.
\hypersetup{pageanchor=false}

% A capa; o parâmetro pode ser "port" ou "eng" para definir a língua
\capaime[port]
%\capaime[eng]

% Se você não quiser usar a capa padrão, você pode criar uma outra
% capa manualmente ou em um programa diferente. No segundo caso, é só
% importar a capa como uma página adicional usando o pacote pdfpages.
%\includepdf{./arquivo_da_capa.pdf}

% A página de rosto da versão para depósito (ou seja, a versão final
% antes da defesa) deve ser diferente da página de rosto da versão
% definitiva (ou seja, a versão final após a incorporação das sugestões
% da banca). Os parâmetros podem ser "port/eng" para a língua e
% "provisoria/definitiva" para o tipo de página de rosto.
%\pagrostoime[port]{definitiva}
\pagrostoime[port]{provisoria}
%\pagrostoime[eng]{definitiva}
%\pagrostoime[eng]{provisoria}

%%%%%%%%%%%%%%%%%%%% DEDICATÓRIA, RESUMO, AGRADECIMENTOS %%%%%%%%%%%%%%%%%%%%%%%

% A definição deste ambiente está no pacote imeusp; se você não
% carregar esse pacote, precisa cuidar desta página manualmente.
\begin{dedicatoria}
Esta seção é opcional e fica numa página separada; ela pode ser usada para
uma dedicatória ou epígrafe.
\end{dedicatoria}

% Após a capa e as páginas de rosto, começamos a numerar as páginas; com isso,
% podemos também reabilitar links para números de páginas no pacote hyperref.
% Isso porque, embora contagem de páginas aqui começe em 1 e no primeiro
% capítulo também, o fato de uma numeração usar algarismos romanos e a outra
% algarismos arábicos é suficiente para evitar problemas.
\pagenumbering{roman}
\hypersetup{pageanchor=true}

% Agradecimentos:
% Se o candidato não quer fazer agradecimentos, deve simplesmente eliminar
% esta página. A epígrafe, obviamente, é opcional; é possível colocar
% epígrafes em todos os capítulos. O comando "\chapter*" faz esta seção
% não ser incluída no sumário.
\chapter*{Agradecimentos}
\epigrafe{Se eu vi mais longe, foi por estar sobre ombros de gigantes.}{Isaac Newton}

Texto texto texto texto texto texto texto texto texto texto texto texto texto
texto texto texto texto texto texto texto texto texto texto texto texto texto
texto texto texto texto texto texto texto texto texto texto texto texto texto
texto texto texto texto. Texto opcional.

% O resumo é obrigatório, em português e inglês. Este comando também gera
% automaticamente a referência para o próprio documento, conforme as normas
% sugeridas da USP
\begin{resumo}{port}
Sistemas Operacionais (SO) são constituídos de um conjunto de abstrações com
o objetivo de fornecer formas de gerenciamento de processos, memória,
armazenamento e dispositivos. O conceito de SO evoluem pela adição ou remoção
de novas características que busquem atender as mudanças de requisitos de
segurança, desempenho, suporte de hardware e modelo de programação. Nas últimas
décadas, muitos pesquisadores se dedicaram para melhorar a abstração de
processos adicionando camadas extras de segurança, melhorando o desempenho, ou
dando suporte para novos recursos de hardware.
%TODO
The endeavours to expand OS capabilities in the process abstraction
open a new and yet to be explored path of research. This paper surveys
state-of-the-art research related to OS process features. We analysed X OS-related works
with a focus on proposed new features and implementation, grouped them into
three categories, and derived a set of relevant characteristics to guide the
development of next-generation process abstractions. Furthermore, we suggest a
collection of microbenchmarks that may help reveal the impact of a new OS
process feature. Lastly, we discuss user applications that
would benefit from capabilities provided by new process
abstractions. Demonstrating and validating tangible benefits for user applications
is essential to assess the advantages and  limitations brought by OS
design changes. This survey will assist in understanding the potential of new
process abstractions and explore how and why new process features could benefit
current OSes.
\end{resumo}

% O resumo é obrigatório, em português e inglês. Este comando também gera
% automaticamente a referência para o próprio documento, conforme as normas
% sugeridas da USP
\begin{resumo}{eng}
Operating Systems (OS) are built upon a set of abstractions to provide process,
memory, storage, and device management. OS concepts evolve by
adding or removing features to meet changing requirements in security, performance,
hardware support, and programming models. Over the last decade many researchers
explored alternatives to advance OS design; in particular, several efforts were
applied to improve the \emph{process} abstraction to add extra security
levels, improve process performance, or support new hardware
resources. The endeavours to expand OS capabilities in the process abstraction
open a new and yet to be explored path of research. This paper surveys
state-of-the-art research related to OS process features. We analysed X OS-related works
with a focus on proposed new features and implementation, grouped them into
three categories, and derived a set of relevant characteristics to guide the
development of next-generation process abstractions. Furthermore, we suggest a
collection of microbenchmarks that may help reveal the impact of a new OS
process feature. Lastly, we discuss user applications that
would benefit from capabilities provided by new process
abstractions. Demonstrating and validating tangible benefits for user applications
is essential to assess the advantages and  limitations brought by OS
design changes. This survey will assist in understanding the potential of new
process abstractions and explore how and why new process features could benefit
current OSes.

\end{resumo}

%%%%%%%%%%%%%%%%%%%%%%%%%%% LISTAS DE FIGURAS ETC. %%%%%%%%%%%%%%%%%%%%%%%%%%%%%

% Todas as listas são opcionais; Usando "\chapter*" elas não são incluídas
% no sumário. As listas geradas automaticamente também não são incluídas
% por conta das opções "notlot" e "notlof" que usamos mais acima.

% Listas criadas manualmente
\chapter*{Lista de Abreviaturas}
\begin{tabular}{rl}
  MMP      & Mondrian Memory Protection\\
  PD       & Protection Domain\\
  SO       & Sistema Operacional\\
  SB       & Secure binding \\
  TLB      & Translation Lookaside Buffer \\
  VMM      & Virtual Machine Monitor \\
  MPM      & Multi-Processing Module \\
  APR      & Apache Portable Runtime \\
  GC       & Garbage Collection \\
  VMA      & Virtual Memory AREA\\
  PTE      & Page Table Entry\\
  USP      & Universidade de São Paulo
\end{tabular}

% Normalmente, "\chapter*" faz o novo capítulo iniciar em uma nova página.
% Como cada uma destas listas é muito curta, não faz muito sentido fazer
% isso aqui. "\let\clearpage\relax" é um "truque sujo" para temporariamente
% desabilitar a quebra de página.

%\chapter*{Lista de Símbolos}
%{\let\cleardoublepage\relax \addvspace{55pt plus 15pt minus 15pt} \chapter*{Lista de Símbolos} }
%\begin{tabular}{rl}
%        $\omega$    & Frequência angular\\
%        $\psi$      & Função de análise \emph{wavelet}\\
%        $\Psi$      & Transformada de Fourier de $\psi$\\
%\end{tabular}

% Listas criadas automaticamente
%\listoffigures
{\let\cleardoublepage\relax \addvspace{55pt plus 15pt minus 15pt} \listoffigures }

%\listoftables
{\let\cleardoublepage\relax \addvspace{55pt plus 15pt minus 15pt} \listoftables }

% Sumário (obrigatório)
\tableofcontents

% Referências indiretas ("x", veja "y") para o índice remissivo (opcionais,
% pois o índice é opcional). É comum colocar esses itens no final do documento,
% junto com o comando \printindex, mas em alguns casos isso torna necessário
% executar texindy (ou makeindex) mais de uma vez, então colocar aqui é melhor.
%\index{Inglês|see{Língua estrangeira}}
%\index{Figuras|see{Floats}}
%\index{Tabelas|see{Floats}}
%\index{Código-fonte|see{Floats}}
%\index{Subcaptions|see{Subfiguras}}
%\index{Sublegendas|see{Subfiguras}}
%\index{Equações|see{Modo Matemático}}
%\index{Fórmulas|see{Modo Matemático}}
%\index{Rodapé, notas|see{Notas de rodapé}}
%\index{Captions|see{Legendas}}
%\index{Versão original|see{Tese/Dissertação, versões}}
%\index{Versão corrigida|see{Tese/Dissertação, versões}}
%\index{Palavras estrangeiras|see{Língua estrangeira}}
%\index{Floats!Algoritmo|see{Floats, Ordem}}
