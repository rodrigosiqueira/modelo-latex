\chapter{Usando o \LaTeX{} e este modelo}

Não é necessário que o texto seja redigido usando \LaTeX{}, mas é fortemente
recomendado o uso dessa ferramenta, pois ela facilita diversas etapas do
trabalho e o resultado final é muito bom\footnote{O uso de um sistema de
controle de versões, como mercurial (\url{mercurial-scm.org}) ou git
(\url{git-scm.com}), também é altamente recomendado.}. Este modelo inclui
vários comentários explicativos e pacotes interessantes para auxiliá-lo com
ele, sendo composto dos arquivos principais de cada exemplo
(\texttt{tese-exemplo.tex}, \texttt{apresentacao-exemplo.tex} e
\texttt{poster-exemplo.tex}) e de vários arquivos auxiliares:

\begin{itemize}
  \item Arquivos com o conteúdo do trabalho:
  \begin{itemize}
    \item \texttt{conteudo/metadados-tese.tex} (orientador, banca etc.)
    \item \texttt{conteudo/folhas-de-rosto.tex} (resumo, dedicatória etc.)
    \item \texttt{conteudo/capitulos.tex}, \texttt{conteudo/apendices.tex},
          \texttt{conteudo/anexos.tex} e demais arquivos carregados por eles
          %(\texttt{cap-*.tex}, \texttt{ape-conjuntos.tex}, \texttt{ann-livre.tex})
    \item \texttt{bibliografia.bib} (dados bibliográficos)
  \end{itemize}

  \item Arquivos com as \textit{packages} usadas e suas configurações (leia
        os comentários neles se quiser modificar algum aspecto do
        documento ou acrescentar alguma \textit{package}):
  \begin{itemize}
    \item \texttt{extras/basics.tex} (\textit{packages} e configurações essenciais),
          \texttt{extras/fonts.tex} (definição das fontes do documento) e
          \texttt{extras/floats.tex} (configurações e melhorias para \textit{floats})
    \item \texttt{extras/thesis-formatting.tex} (aparência: espaçamento, sumário etc.)
    \item \texttt{extras/index.tex} (configuração do índice remissivo)
    \item \texttt{extras/hyperlinks.tex} (configuração das referências cruzadas)
    \item \texttt{extras/source-code.tex} (exibição de código-fonte e pseudocódigo)
    \item \texttt{extras/utils.tex} (\textit{packages} adicionais diversas)
    \item \texttt{extras/bibconfig.tex} (configuração da bibliografia)
  \end{itemize}

  \item Outros arquivos auxiliares (geralmente não precisam ser editados):
  \begin{itemize}
    \item \texttt{extras/imeusp-capa.sty} (formatação da capa e demais páginas iniciais)
    \item \texttt{extras/imeusp-headers.sty} (formatação dos cabeçalhos)
    \item \texttt{extras/lstpseudocode.sty} (suporte a pseudocódigo com \textsf{listings})
    \item \texttt{extras/annex.sty} (permite adicionar anexos) e
          \texttt{extras/appendixlabel.sty} (melhora a lista de
          apêndices/anexos no sumário)
    \item \texttt{extras/beamer*.sty} (\textit{layouts} e cores para
          apresentações e \textit{posters})
    \item \texttt{extras/plainnat-ime.*} (estilo plainnat para bibliografias)\index{biblatex}
    \item \texttt{extras/alpha-ime.bst} (estilo alpha para bibliografias com
          bibtex)\index{bibtex}
    \item \texttt{extras/natbib-ime.sty} (tradução da \textit{package}
          padrão natbib)\index{natbib}
    \item \texttt{hyperxindy.xdy} (configuração para xindy) e
          \texttt{mkidxhead.ist} (configuração para makeindex)
    \item \texttt{latexmkrc} e \texttt{Makefile} (automatizam a geração do
          documento com os comandos \textsf{latexmk} e \textsf{make} respectivamente)
  \end{itemize}
\end{itemize}

Para compilar o documento, basta executar o comando \textsf{latexmk} (ou
\textsf{make})\footnote{Você também pode usar \textsf{latexmk poster-exemplo}
e \textsf{latexmk apresentacao-exemplo}.}. Talvez seu editor ofereça uma
opção de menu para compilar o documento, mas ele provavelmente depende do
\textsf{latexmk} para isso. \LaTeX{} gera diversos arquivos auxiliares
durante a compilação que, em algumas raras situações, podem ficar
inconsistentes (causando erros de compilação ou erros no PDF gerado, como
referências faltando ou numeração de páginas incorreta no sumário). Nesse
caso, é só usar o comando \textsf{latexmk -C} (ou \textsf{make clean}),
que apaga todos os arquivos auxiliares gerados, e em seguida rodar
\textsf{latexmk} (ou \textsf{make}) novamente.

\section{Instalação do \LaTeX{}}

\LaTeX{} é, na verdade, um conjunto de programas. Ao invés de procurar e
baixar cada um deles, o mais comum é baixar um pacote com todos eles juntos.
Há dois pacotes desse tipo disponíveis: MiK\TeX{} (\url{miktex.org}) e
\TeX{}Live (\url{www.tug.org/texlive}). Ambos funcionam em Linux, Windows e
MacOS X. Em Linux, \TeX{}Live costuma estar disponível para instalação junto
com os demais opcionais do sistema. Em MacOS X, o mais popular é o Mac\TeX{}
(\url{www.tug.org/mactex/}), a versão do \TeX{}Live para MacOS X.  Em Windows,
o mais comumente usado é o MiK\TeX{}.

Por padrão, eles não instalam tudo que está disponível, mas sim apenas os
componentes mais usados, e oferecem um gestor de pacotes que permite adicionar
outros. Embora uma instalação completa do \LaTeX{} seja relativamente grande
(perto de 5GB), em geral vale a pena instalar a maior parte dos pacotes. Se
você preferir uma instalação mais ``enxuta'', não deixe de incluir todos os
pacotes necessários para este modelo, como indicado no arquivo README.md.

Também é muito importante ter o \textsf{latexmk} (ou o \textsf{make}). No Linux,
a instalação é similar à de outros programas. No MacOS X e no Windows,
\textsf{latexmk} pode ser instalado pelo gestor de pacotes do MiK\TeX{} ou
\TeX{}Live. Observe que ele depende da linguagem \textsf{perl}, que precisa ser
instalada à parte no Windows (\url{www.perl.org/get.html}).

\section{Bibliografia}

Sugerimos que você faça referências bibliográficas nos formatos ``alpha'' ou
``plainnat''.  Se estiver usando natbib+bibtex\index{natbib}\index{bibtex},
use os arquivos .bst ``alpha-ime.bst'' ou ``plainnat-ime.bst'', que são
versões desses dois formatos traduzidas para o português. Se estiver usando
biblatex\index{biblatex} (recomendado), escolha o estilo ``alphabetic''
(que é um dos estilos padrão do biblatex) ou ``plainnat-ime''. O arquivo de
exemplo inclui todas essas opções; basta des-comentar as linhas
correspondentes e, se necessário, modificar o arquivo Makefile para chamar
o bibtex\index{bibtex} ao invés do biber\index{biber} (este último é usado
em conjunto com o biblatex).

\section{Perguntas Frequentes sobre o Modelo}

\begin{itemize}

\item \textbf{Posso usar pacotes \LaTeX{} adicionais aos sugeridos?}\\
Com certeza! Você pode modificar o arquivo o quanto desejar, o modelo serve só como uma ajuda inicial para o seu trabalho. Observe, no entanto, que ele é baseado na classe \textsf{book} e deve funcionar, com alterações mínimas, com \textsf{article}; já as classes \textsf{KOMA-Script} ou a classe \textsf{memoir} podem ser mais difíceis de adaptar. Além disso, \textsf{pstricks} e \textit{packages} derivadas dependem da linguagem PostScript e, portanto, não funcionam facilmente com as versões modernas de \LaTeX{}.

\item \textbf{As figuras podem ser colocadas no meio do texto ou devem ficar no final dos capítulos?}\\
Em geral, as figuras devem ser apresentadas assim que forem referenciadas. Colocá-las no final dos capítulos dificultaria um pouco a leitura, mas isso depende do estilo do autor, orientador ou lugar de publicação. Converse com seu orientador!

\item \textbf{As figuras e tabelas são colocadas em lugares ruins.}\\
Veja a discussão a respeito na Seção~\ref{sec:limitations}.

\item \textbf{Estou tendo problemas com caracteres acentuados!}\\
Veja a discussão a respeito na Seção~\ref{sec:limitations}.

\item \textbf{Existe algo específico para citações de páginas web?}\\
Biblatex define o tipo ``online'', que deve ser usado para materiais com título, autor etc., como uma postagem ou comentário em um blog, um gráfico ou mesmo uma mensagem de email para uma lista de discussão. Bibtex\index{bibtex}, por padrão, não tem um tipo específico para isso; com ele, normalmente usa-se o campo ``howpublished'' para especificar que se trata de um recurso \textit{online}. Se o que você está citando não é algo determinado com título, autor etc. mas sim um sítio (como uma empresa ou um produto), pode ser mais adequado colocar a referência apenas como nota de rodapé e não na lista de referências; nesses casos, algumas pessoas acrescentam uma segunda lista de referências especificamente para recursos \textit{online} (biblatex\index{biblatex} permite criar múltiplas bibliografias). Já artigos disponíveis \textit{online} mas que fazem parte de uma publicação de formato tradicional (mesmo que apenas \textit{online}), como os anais de um congresso, devem ser citados por seu tipo verdadeiro e apenas incluir o campo ``url'' (não é nem necessário usar o comando \textsf{\textbackslash{}url\{\}}), aceito por todos os tipos de documento do bibtex/biblatex.

\item \textbf{A bibliografia está sendo impressa em inglês (usa ``and'' ao invés de ``e'' para separar os nomes dos autores).}\\
Você deve estar usando um estilo de bibliografia bibtex diferente dos sugeridos. Uma simples solução é copiar o arquivo de estilo correspondente da sua instalação \LaTeX{} para o diretório onde seus arquivos estão e mudar ``and'' por ``e'' (ou ``\&'' se preferir) na função format.names. O mais recomendado, no entanto, é usar biblatex: ele é mais fácil de adaptar para diferentes estilos, tem pleno suporte a diferentes línguas e é possível personalizar as traduções (há um exemplo no modelo).

\item \textbf{Aparece uma folha em branco entre os capítulos}\\
Essa característica foi colocada propositalmente, dado que todo capítulo deve (ou deveria) começar em uma página de numeração ímpar (lado direito do documento). Se quiser mudar esse comportamento, acrescente ``openany'' como opção da classe, i.e., \textsf{\textbackslash{}documentclass[openany,11pt,twoside,a4paper]\{book\}}.

\item \textbf{É possível resumir o nome das seções/capítulos que aparece no topo das páginas e no sumário?}\\
Sim, usando a sintaxe \textsf{\textbackslash{}section[mini-titulo]\{titulo enorme\}}. Isso é especialmente útil nos \textit{captions}\index{Legendas} das figuras e tabelas, que muitas vezes são demasiadamente longos para a lista de figuras/tabelas.

\item \textbf{Existe algum programa para gerenciar referências em formato bibtex?}\\
Sim, há vários. Uma opção bem comum é o JabRef; outra é usar Zotero\index{Zotero} ou Mendeley\index{Mendeley} e exportar os dados deles no formato .bib.

\item \textbf{Como faço para usar o Makefile (comando make) no Windows?}\\
Lembre-se que a ferramenta recomendada para compilação do documento é o \textsf{latexmk}, então você não precisa do \textsf{make}. Mas, se quiser usá-lo, você pode instalar o MSYS2 (\url{www.msys2.org}) ou o Windows Subsystem for Linux (procure as versões de Linux disponíveis na Microsoft Store). Se você pretende usar algum dos editores sugeridos, é possível deixar a compilação a cargo deles, também dispensando o \textsf{make}.

\item \textbf{Como eu faço para...}\\
Leia os comentários dos arquivos ``tese-exemplo.tex'' e outros que compõem este modelo, além do tutorial (Capítulo \ref{chap:tutorial}) e dos exemplos do Capítulo \ref{chap:exemplos}; é provável que haja uma dica neles ou, pelo menos, a indicação da \textit{package} relacionada ao que você precisa.

\end{itemize}
